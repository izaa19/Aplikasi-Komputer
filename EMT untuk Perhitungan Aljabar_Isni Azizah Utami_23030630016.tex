% Options for packages loaded elsewhere
\PassOptionsToPackage{unicode}{hyperref}
\PassOptionsToPackage{hyphens}{url}
\documentclass[
]{book}
\usepackage{xcolor}
\usepackage{amsmath,amssymb}
\setcounter{secnumdepth}{-\maxdimen} % remove section numbering
\usepackage{iftex}
\ifPDFTeX
  \usepackage[T1]{fontenc}
  \usepackage[utf8]{inputenc}
  \usepackage{textcomp} % provide euro and other symbols
\else % if luatex or xetex
  \usepackage{unicode-math} % this also loads fontspec
  \defaultfontfeatures{Scale=MatchLowercase}
  \defaultfontfeatures[\rmfamily]{Ligatures=TeX,Scale=1}
\fi
\usepackage{lmodern}
\ifPDFTeX\else
  % xetex/luatex font selection
\fi
% Use upquote if available, for straight quotes in verbatim environments
\IfFileExists{upquote.sty}{\usepackage{upquote}}{}
\IfFileExists{microtype.sty}{% use microtype if available
  \usepackage[]{microtype}
  \UseMicrotypeSet[protrusion]{basicmath} % disable protrusion for tt fonts
}{}
\makeatletter
\@ifundefined{KOMAClassName}{% if non-KOMA class
  \IfFileExists{parskip.sty}{%
    \usepackage{parskip}
  }{% else
    \setlength{\parindent}{0pt}
    \setlength{\parskip}{6pt plus 2pt minus 1pt}}
}{% if KOMA class
  \KOMAoptions{parskip=half}}
\makeatother
\usepackage{graphicx}
\makeatletter
\newsavebox\pandoc@box
\newcommand*\pandocbounded[1]{% scales image to fit in text height/width
  \sbox\pandoc@box{#1}%
  \Gscale@div\@tempa{\textheight}{\dimexpr\ht\pandoc@box+\dp\pandoc@box\relax}%
  \Gscale@div\@tempb{\linewidth}{\wd\pandoc@box}%
  \ifdim\@tempb\p@<\@tempa\p@\let\@tempa\@tempb\fi% select the smaller of both
  \ifdim\@tempa\p@<\p@\scalebox{\@tempa}{\usebox\pandoc@box}%
  \else\usebox{\pandoc@box}%
  \fi%
}
% Set default figure placement to htbp
\def\fps@figure{htbp}
\makeatother
\setlength{\emergencystretch}{3em} % prevent overfull lines
\providecommand{\tightlist}{%
  \setlength{\itemsep}{0pt}\setlength{\parskip}{0pt}}
\usepackage{bookmark}
\IfFileExists{xurl.sty}{\usepackage{xurl}}{} % add URL line breaks if available
\urlstyle{same}
\hypersetup{
  hidelinks,
  pdfcreator={LaTeX via pandoc}}

\author{}
\date{}

\begin{document}
\frontmatter

\mainmatter
\chapter{EMT untuk Perhitungan Aljabar}\label{emt-untuk-perhitungan-aljabar}

Nama: Isni Azizah Utami

Kelas Matematika B

NIM : 23030630016

Pada notebook ini Anda belajar menggunakan EMT untuk melakukan berbagai perhitungan terkait dengan materi atau topik dalam Aljabar. Kegiatan yang harus Anda lakukan adalah sebagai berikut:

\begin{enumerate}
\def\labelenumi{\arabic{enumi}.}
\tightlist
\item
  Membaca secara cermat dan teliti notebook ini Menerjemahkan teks bahasa Inggris ke bahasa Indonesia;
\item
  Mencoba contoh-contoh perhitungan (perintah EMT) dengan cara meng-ENTER setiap perintah EMT yang ada (pindahkan kursor ke baris perintah)
\item
  Jika perlu Anda dapat memodifikasi perintah yang ada dan memberikan keterangan / penjelasan tambahan terkait hasilnya.
\item
  Menyisipkan baris-baris perintah baru untuk mengerjakan soal-soal Aljabar dari file PDF yang saya berikan;
\item
  Memberi catatan hasilnya.
\item
  Jika perlu tuliskan soalnya pada teks notebook (menggunakan format LaTeX).
\item
  Gunakan tampilan hasil semua perhitungan yang eksak atau simbolik dengan format LaTeX. (Seperti contoh-contoh pada notebook ini.)
\end{enumerate}

\chapter{Contoh pertama}\label{contoh-pertama}

Menyederhanakan bentuk aljabar: \[6x^{-3}y^5\times -7x^2y^{-9}\]\textgreater\$\&6*x\textsuperscript{(-3)*y}5*-7*x\textsuperscript{2*y}(-9)

\[-\frac{42}{x\,y^4}\]Menjabarkan: \[(6x^{-3}+y^5)(-7x^2-y^{-9})\] \textgreater\$\&showev('expand((6*x\textsuperscript{(-3)+y}5)*(-7*x\textsuperscript{2-y}(-9))))

\[{\it expand}\left(\left(-\frac{1}{y^9}-7\,x^2\right)\,\left(y^5+  \frac{6}{x^3}\right)\right)=-7\,x^2\,y^5-\frac{1}{y^4}-\frac{6}{x^3  \,y^9}-\frac{42}{x}\]

\chapter{Baris Perintah}\label{baris-perintah}

Baris perintah Euler terdiri dari satu atau beberapa perintah Euler diikuti dengan titik koma ``;'' atau koma ``,''. Titik koma mencegah pencetakan hasil. Koma setelah perintah terakhir dapat dihilangkan. Baris perintah berikut hanya akan mencetak hasil ekspresi, bukan tugas atau perintah format.

\textgreater r:=2; h:=4; pi*r\^{}2*h/3

\begin{verbatim}
16.7551608191
\end{verbatim}

Perintah harus dipisahkan dengan yang kosong. Baris perintah berikut mencetak dua hasilnya.

\textgreater pi*2*r*h, \%+2*pi*r*h

Ingat tanda \% menyatakan hasil perhitungan terakhir sebelumnya

\begin{verbatim}
50.2654824574
100.530964915
\end{verbatim}

Baris perintah dieksekusi dalam urutan yang ditekan pengguna kembali. Jadi Anda mendapatkan nilai baru setiap kali Anda menjalankan baris kedua.

\textgreater x := 1; \textgreater x := cos(x)

nilai cosinus (x dalam radian)

\begin{verbatim}
0.540302305868
\end{verbatim}

\textgreater x := cos(x)

\begin{verbatim}
0.857553215846
\end{verbatim}

Jika dua garis terhubung dengan ``\ldots{}'' kedua garis akan selalu dieksekusi secara bersamaan.

\textgreater x := 1.5; \ldots{}\\
\textgreater{} x := (x+2/x)/2, x := (x+2/x)/2, x := (x+2/x)/2,

\begin{verbatim}
1.41666666667
1.41421568627
1.41421356237
\end{verbatim}

Ini juga merupakan cara yang baik untuk menyebarkan long command pada dua atau lebih baris. Anda dapat menekan Ctrl+Return untuk membagi garis menjadi dua pada posisi kursor saat ini, atau Ctrl+Back untuk menggabungkan garis.

Sedangkan untuk fold semua multi-garis tekan Ctrl + L. Kemudian garis-garis berikutnya hanya akan terlihat, jika salah satunya memiliki fokus. Untuk fold satu multi-baris, mulailah baris pertama dengan ``\%+''.

\textgreater\%+ x=4+5; \ldots{}\\
\textgreater{} // This line will not be visible once the cursor is off the line

A line starting with \%\% will be completely invisible.

\begin{verbatim}
81
\end{verbatim}

Euler Math Toolbox mendukung loop di baris perintah, selama mereka masuk ke dalam satu baris atau multi-baris. Dalam program, pembatasan ini tidak berlaku, tentu saja. Untuk informasi lebih lanjut lihat pengantar berikut.

\textgreater x=1; for i=1 to 5; x := (x+2/x)/2, end;

menghitung akar 2

\begin{verbatim}
1.5
1.41666666667
1.41421568627
1.41421356237
1.41421356237
\end{verbatim}

Tidak apa-apa untuk menggunakan multi-line. Pastikan baris diakhiri dengan ``\ldots{}''.

\textgreater x := 1.5; // tambahkan komentar disini seblum \ldots{}\\
\textgreater{} repeat xnew:=(x+2/x)/2; until xnew\textasciitilde=x; \ldots{}\\
\textgreater{} x := xnew; \ldots{}\\
\textgreater{} end; \ldots{}\\
\textgreater{} x,

\begin{verbatim}
1.41421356237
\end{verbatim}

Struktur bersyarat juga berfungsi.

\textgreater if E\textsuperscript{pi\textgreater pi}E; then ``Thought so!'', endif;

\begin{verbatim}
Thought so!
\end{verbatim}

Saat Anda menjalankan perintah, kursor dapat berada di posisi mana pun di baris perintah. Anda dapat kembali ke perintah sebelumnya atau melompat ke perintah berikutnya dengan tombol panah. Atau Anda dapat mengklik ke bagian komentar di atas perintah untuk menuju ke perintah.

Saat Anda menggerakkan kursor di sepanjang garis, pasangan tanda kurung atau kurung buka dan tutup akan disorot. Dan juga, perhatikan baris status. Setelah kurung buka fungsi sqrt(), baris status akan menampilkan teks bantuan untuk fungsi tersebut. Jalankan perintah dengan tombol kembali.

\textgreater sqrt(sin(10°)/cos(20°))

\begin{verbatim}
0.429875017772
\end{verbatim}

Untuk melihat bantuan untuk perintah terbaru, buka jendela bantuan dengan F1. Di sana, Anda dapat memasukkan teks untuk dicari. Pada baris kosong, bantuan untuk jendela bantuan akan ditampilkan. Anda dapat menekan escape untuk menghapus garis, atau untuk menutup jendela bantuan.

Anda dapat mengklik dua kali pada perintah apa pun untuk membuka bantuan untuk perintah ini. Coba klik dua kali perintah exp di bawah ini di baris perintah.

\textgreater exp(log(2.5))

\begin{verbatim}
2.5
\end{verbatim}

Anda juga dapat menyalin dan menempel di Euler. Gunakan Ctrl-C dan Ctrl-V untuk ini. Untuk menandai teks, seret mouse atau gunakan shift bersamaan dengan tombol kursor. Selain itu, Anda dapat menyalin tanda kurung yang disorot.

\chapter{Sintak Dasar}\label{sintak-dasar}

Euler Math Toolbox tahu fungsi matematika yang biasa digunakan. Seperti yang Anda lihat di atas, fungsi trigonometri bekerja dalam radian atau derajat. Untuk mengonversi ke derajat, tambahkan simbol derajat (dengan tombol F7) ke dalam nilainya, atau gunakan fungsi rad(x). Fungsi akar kuadrat disebut sqrt dalam Euler. Tentu saja, x\^{}(1/2) juga memungkinkan.

Untuk menyetel variabel, gunakan ``='' atau ``:=''. Demi kejelasan, pengantar ini menggunakan bentuk yang terakhir/terbaru. Spasi tidak menjadi masalah. Tetapi spasi di antara perintah sangat diharapkan.

Beberapa perintah dalam satu baris dipisahkan dengan ``,'' atau ``;''. Titik koma menekan output dari perintah. Di akhir baris perintah ``,'' diasumsikan, jika ``;'' hilang.

\textgreater g:=9.81; t:=2.5; 1/2*g*t\^{}2

\begin{verbatim}
30.65625
\end{verbatim}

EMT menggunakan sintaks pemrograman untuk ekspresi. Untuk memasukkan \[e^2 \cdot \left( \frac{1}{3+4 \log(0.6)}+\frac{1}{7} \right)\]Anda harus mengatur tanda kurung dengan benar dan menggunakan ``/'' untuk pecahan. Perhatikan tanda kurung yang disorot untuk bantuan. Perhatikan bahwa konstanta Euler e diberi nama E dalam EMT.

\textgreater E\^{}2*(1/(3+4*log(0.6))+1/7)

\begin{verbatim}
8.77908249441
\end{verbatim}

Untuk menghitung ekspresi rumit seperti \[\left(\frac{\frac17 + \frac18 + 2}{\frac13 + \frac12}\right)^2 \pi\]anda harus memasukkannya dalam bentuk baris.

\textgreater((1/7 + 1/8 + 2) / (1/3 + 1/2))\^{}2 * pi

\begin{verbatim}
23.2671801626
\end{verbatim}

Letakkan tanda kurung dengan hati-hati di sekitar sub-ekspresi yang perlu dihitung terlebih dahulu. EMT membantu Anda dengan menyorot ekspresi bahwa braket penutup selesai. Anda juga harus memasukkan nama ``pi'' untuk huruf Yunani pi.

Hasil dari perhitungan ini adalah bilangan floating point. Secara default dicetak dengan akurasi sekitar 12 digit. Di baris perintah berikut, kita juga belajar bagaimana kita bisa merujuk ke hasil sebelumnya dalam baris yang sama.

\textgreater1/3+1/7, fraction \%

\begin{verbatim}
0.47619047619
10/21
\end{verbatim}

Perintah Euler dapat berupa ekspresi atau perintah primitif. Ekspresi terbuat dari operator dan fungsi. Jika diperlukan, hal tersebut harus berisi tanda kurung untuk memaksa urutan eksekusi yang benar. Jika ragu, memasang braket atau tanda kurung adalah ide yang bagus. Perhatikan bahwa EMT menunjukkan tanda kurung buka dan tutup saat mengedit baris perintah.

\textgreater(cos(pi/4)+1)\textsuperscript{3*(sin(pi/4)+1)}2

\begin{verbatim}
14.4978445072
\end{verbatim}

Operator numerik Euler meliputi

\begin{enumerate}
\def\labelenumi{\arabic{enumi}.}
\tightlist
\item
  \begin{itemize}
  \tightlist
  \item
    unary or operator plus\\
  \end{itemize}
\item
  \begin{itemize}
  \tightlist
  \item
    unary or operator minus
  \end{itemize}
\item
  \begin{itemize}
  \tightlist
  \item
    operator perkalian\\
  \end{itemize}
\item
  / operator pecahan\\
\item
  . produk matriks\\
\item
  a\^{}b pangkat untuk a positif atau bilangan bulat b (a**b juga berfungsi)
\item
  n! operator faktorial
\item
  dan masih banyak lagi.
\end{enumerate}

Berikut adalah beberapa fungsi yang mungkin Anda butuhkan. Ada banyak lagi. sin,cos,tan,atan,asin,acos,rad,deg\\
log,exp,log10,sqrt,logbase\\
bin,logbin,logfac,mod,floor,ceil,round,abs,sign\\
conj,re,im,arg,conj,real,complex\\
beta,betai,gamma,complexgamma,ellrf,ellf,ellrd,elle\\
bitand,bitor,bitxor,bitnot

Beberapa perintah memiliki alias, misalnya ln untuk log.

\textgreater ln(E\^{}2), arctan(tan(0.5))

\begin{verbatim}
2
0.5
\end{verbatim}

\textgreater sin(30°)

\begin{verbatim}
0.5
\end{verbatim}

Pastikan untuk menggunakan tanda kurung (kurung bulat), setiap kali ada keraguan tentang urutan eksekusi! Berikut ini tidak sama dengan (2\textsuperscript{3)}4, yang merupakan default untuk 2\textsuperscript{3}4 di EMT (beberapa sistem numerik melakukannya dengan cara lain).

\textgreater2\textsuperscript{3}4, (2\textsuperscript{3)}4, 2\textsuperscript{(3}4)

\begin{verbatim}
2.41785163923e+24
4096
2.41785163923e+24
\end{verbatim}

\chapter{Bilangan Asli}\label{bilangan-asli}

Tipe data utama dalam Euler adalah bilangan real. Real direpresentasikan dalam format IEEE dengan akurasi sekitar 16 digit desimal.

\textgreater longest 1/3

\begin{verbatim}
     0.3333333333333333 
\end{verbatim}

Representasi ganda internal membutuhkan 8 byte.

Representasi ganda adalah format penyimpanan untuk floating-point yang menggunakan 64 bit(8 byte).

\textgreater printdual(1/3)

\begin{verbatim}
1.0101010101010101010101010101010101010101010101010101*2^-2
\end{verbatim}

\textgreater printhex(1/3)

\begin{verbatim}
5.5555555555554*16^-1
\end{verbatim}

Perbedaan `printdual' dan `printhex' adalah `printdual' yakni mencetak representasi internal dari sebuah bilangan floating-point dalam format presisi ganda (pendekatan yang sangat dekat dengan nilai aslinya tetapi tidak persis sama.) meskipun ia tergantung pada konteks bahasa pemograman tertentu. sedangkan `printhex' yakni representasi dari nilai floating-point dalam bentuk heksadesimal(basis 16), heksadesimal ini adalah cara yang lebih ringkas untuk menampilkan nilai biner karena setiap digit heksadesimal mempresentasikan empat digit biner.

\chapter{Strings}\label{strings}

Sebuah string dalam Euler didefinisikan dengan ``\ldots{}''

\textgreater{}``A string can contain anything.''

\begin{verbatim}
A string can contain anything.
\end{verbatim}

String dapat digabungkan dengan \textbar{} atau dengan +. Ini juga berfungsi dengan angka, yang dikonversi menjadi string dalam kasus itu.

\textgreater{}``The area of the circle with radius'' + 2 + '' cm is '' + pi*4 + '' cm\^{}2.''

\begin{verbatim}
The area of the circle with radius 2 cm is 12.5663706144 cm^2.
\end{verbatim}

Pada String fungsi print mengonversi angka menjadi string. Ini dapat mengambil sejumlah digit dan sejumlah tempat (0 untuk keluaran padat), dan secara optimal satu unit

\textgreater{}``Golden Ratio :'' + print((1+sqrt(5))/2,5,0)

\begin{verbatim}
Golden Ratio : 1.61803
\end{verbatim}

Ada string khusus `none', yang tidak mencetak. Dikembalikan oleh beberapa fungsi, ketika hasilnya tidak penting. (Dikembalikan secara otomatis, jika fungsi tidak memiliki pernyataan pengembalian).

\textgreater none

Untuk mengonversi string menjadi angka, cukup mengevaluasinya. Ini bekerja untuk ekspresi juga (lihat dibawah).

\textgreater{}``1234.5''()

\begin{verbatim}
1234.5
\end{verbatim}

Untuk mendefinisikan vektor string, gunakan notasi vektor {[}\ldots{]}

\textgreater v:={[}``affe'',``charlie'',``bravo''{]}

\begin{verbatim}
affe
charlie
bravo
\end{verbatim}

Vektor string kosong dilambangkan dengan {[}none{]}. Dan vektor string dapat digabungkan dengan `\textbar{}'.

\textgreater w:={[}none{]}; w\textbar v\textbar v

\begin{verbatim}
affe
charlie
bravo
affe
charlie
bravo
\end{verbatim}

String dapat berisi karakter Unicode. Secara internal, string ini berisi kode UTF-8. untuk menghasilkan string seperti itu, gunakan u''\ldots'' dan salah satu entitas HTML.

String Unicode dapat digabungkan seperti string lainnya.

\textgreater u''α = '' + 45 + u''°'' // pdfLaTeX mungkin gagal menampilkan secara benar

\begin{verbatim}
α = 45°
\end{verbatim}

Dalam komentar, entitas yang sama seperti α, β etc. dapat digunakan. Ini mungkin merupakan alternatif cepat untuk Lateks. (Detail lebih lanjut pada komentar di bawah).

Ada beberapa fungsi untuk membuat atau menganalisis string unicode.

Fungsi strtochsr() akan mengenali string Unicode, dan menerjemahkannya dengan benar.

\textgreater v=strtochar(u''Ä is a German letter'')

\begin{verbatim}
[196,  32,  105,  115,  32,  97,  32,  71,  101,  114,  109,  97,  110,
32,  108,  101,  116,  116,  101,  114]
\end{verbatim}

Perintah ini menghasilkan array atau daftar angka berupa vektor angka yang mewakili karakter dalam string dalam bentuk kode Unicode.

Fungsi kebalikannya adalah chartoutf().

\textgreater v{[}1{]}=strtochar(u''Ü``){[}1{]}; chartoutf(v)

\begin{verbatim}
Ü is a German letter
\end{verbatim}

Fungsi utf()dapat menerjemahkan string dengan entitas dalam variabel menjadi string Unicode.

\textgreater s=``We have α=β.''; utf(s) // pdfLaTeX mungkin gagal menampilkan secara benar

\begin{verbatim}
We have α=β.
\end{verbatim}

Memungkinkan juga untuk menggunakan entitas numerik.

\textgreater u''Ähnliches''

\begin{verbatim}
Ähnliches
\end{verbatim}

\chapter{Nilai Boolean}\label{nilai-boolean}

Nilai boolean direpresentasikan dengan 1=true atau 0=false dalam euler. String dapat dibandingkan, seperti halnya angka.

\textgreater2\textless1, ``apel''\textless{}``banana''

\begin{verbatim}
0
1
\end{verbatim}

``and'' adalah operator ``\&\&'' dan ``or'' adalah operator ``\textbar\textbar{}'', seperti dalam bahasa C. (kata ``and'' dan ``or'' hanya dapat digunakan dalam kondisi ``if''.)

\textgreater2\textless E \&\& E\textless3

\begin{verbatim}
1
\end{verbatim}

Operator Boolean mematuhi aturan bahasa matriks

\textgreater(1:10)\textgreater5, nonzeros(\%)

\begin{verbatim}
[0,  0,  0,  0,  0,  1,  1,  1,  1,  1]
[6,  7,  8,  9,  10]
\end{verbatim}

Anda dapat menggunakan fungsi nonzeros() untuk mengekstrak elemen tertentu dari sebuah vektor. Pada contoh, kita menggunakan kondisional isprime(n).

\textgreater N=2\textbar3:2:99 // N berisi elemen 2 dan bilangan2 ganjil dari 3 s.d. 99

\begin{verbatim}
[2,  3,  5,  7,  9,  11,  13,  15,  17,  19,  21,  23,  25,  27,  29,
31,  33,  35,  37,  39,  41,  43,  45,  47,  49,  51,  53,  55,  57,
59,  61,  63,  65,  67,  69,  71,  73,  75,  77,  79,  81,  83,  85,
87,  89,  91,  93,  95,  97,  99]
\end{verbatim}

\textgreater N{[}nonzeros(isprime(N)){]} //pilih anggota2 N yang prima

\begin{verbatim}
[2,  3,  5,  7,  11,  13,  17,  19,  23,  29,  31,  37,  41,  43,  47,
53,  59,  61,  67,  71,  73,  79,  83,  89,  97]
\end{verbatim}

\chapter{Output Formats}\label{output-formats}

Default output formats EMT adalah 12 digit. Untuk memastikan yang kita lihat adalah bentuk default, maka perlu direset format.

\textgreater defformat; pi

\begin{verbatim}
3.14159265359
\end{verbatim}

Secara internal, EMT menggunakan standar IEEE (Institute of Electrical and Electronics Engineers) untuk bilangan ganda dengan sekitar 16 digit desimal. Untuk melihat bentuk digit penuh, gunakan perintah ``longestformat'' atau gunakan operator ``longest'' untuk memunculkannya.

\textgreater longest pi

\begin{verbatim}
      3.141592653589793 
\end{verbatim}

Berikut ini adalah repesentasi heksadesimal internal dari bilangan ganda.

\textgreater printhex(pi)

\begin{verbatim}
3.243F6A8885A30*16^0
\end{verbatim}

Format output dapat diubah secara permanen dengan perintah format.

\textgreater format(12,5); 1/3, pi, sin(1)

\begin{verbatim}
    0.33333 
    3.14159 
    0.84147 
\end{verbatim}

Format standarnya adalah 12.

\textgreater format(12); 1/3

\begin{verbatim}
0.333333333333
\end{verbatim}

Fungsi seperti ``shortestformat'', ``shortformat'', ``longformat'' bekerja untuk vektor dengan cara berikut.

\textgreater shortestformat; random(3,8)

\begin{verbatim}
  0.66    0.2   0.89   0.28   0.53   0.31   0.44    0.3 
  0.28   0.88   0.27    0.7   0.22   0.45   0.31   0.91 
  0.19   0.46  0.095    0.6   0.43   0.73   0.47   0.32 
\end{verbatim}

Format standar untuk skalar adalah 12, tetapi ini dapat diubah.

\textgreater setscalarformat(5); pi

\begin{verbatim}
3.1416
\end{verbatim}

Fungsi ``longestformat'' juga menetapkan format skalar.

\textgreater longestformat; pi

\begin{verbatim}
3.141592653589793
\end{verbatim}

Sebagai referensi, berikut ini daftar format output yang paling penting. 1. shortestformat shortformat longformat, longestformat 2. format(length,digits) goodformat(length) 3. fracformat(length) 4. defformat

Akurasi internal EMT adalah sekitar 16 tempat desimal, yang merupakan standar IEEE. Angka disimpan dalam format internal ini.

Tetapi format output EMT dapat diatur dengan cara yang fleksibel.

\textgreater longestformat; pi,

\begin{verbatim}
3.141592653589793
\end{verbatim}

\textgreater format(10,5); pi

\begin{verbatim}
  3.14159 
\end{verbatim}

The default is defformat().

\textgreater defformat; // default

Ada operator pendek yang hanya mencetak satu nilai. Operator ``longest'' akan mencetak semua digit angka yang valid.

\textgreater longest pi\^{}2/2

\begin{verbatim}
      4.934802200544679 
\end{verbatim}

Ada juga operator singkat untuk mencetak hasil dalam format pecahan. Kami sudah menggunakannya di atas.

\textgreater fraction 1+1/2+1/3+1/4

\begin{verbatim}
25/12
\end{verbatim}

Karena format internal menggunakan cara biner untuk menyimpan angka, maka nilai 0,1 tidak akan terwakili dengan tepat. Kesalahan bertambah sedikit, seperti yang Anda lihat dalam perhitungan berikut ini.

\textgreater longest 0.1+0.1+0.1+0.1+0.1+0.1+0.1+0.1+0.1+0.1-1

\begin{verbatim}
 -1.110223024625157e-16 
\end{verbatim}

Tetapi, dengan ``longformat'' default, Anda tidak akan melihat hal ini. Untuk kenyamanan, output angka yang sangat kecil adalah 0.

\textgreater0.1+0.1+0.1+0.1+0.1+0.1+0.1+0.1+0.1+0.1-1

\begin{verbatim}
0
\end{verbatim}

\chapter{Expressions}\label{expressions}

Strings atau names dapat digunakan untuk menyimpan ekspresi matematika, yang dapat dievaluasi oleh EMT. Untuk ini, gunakan tanda kurung setelah ekspresi. Jika Anda bermaksud menggunakan string sebagai ekspresi, gunakan konvensi untuk menamainya ``fx'' atau ``fxy'', dll. Ekspresi lebih diutamakan daripada fungsi.

Variabel global dapat digunakan dalam evaluasi.

\textgreater r:=2; fx:=``pi*r\^{}2''; longest fx()

\begin{verbatim}
      12.56637061435917 
\end{verbatim}

Parameter ditetapkan ke x, y, dan z dalam urutan tersebut. Parameter tambahan dapat ditambahkan dengan menggunakan parameter yang ditetapkan.

\textgreater fx:=``a*sin(x)\^{}2''; fx(5,a=-1)

\begin{verbatim}
-0.919535764538
\end{verbatim}

Perhatikan bahwa ekspresi akan selalu menggunakan variabel global, meskipun ada variabel dalam fungsi dengan nama yang sama. (Jika tidak, evaluasi ekspresi dalam fungsi dapat memberikan hasil yang sangat membingungkan bagi pengguna yang memanggil fungsi tersebut).

\textgreater at:=4; function f(expr,x,at) := expr(x); \ldots{}

\textgreater{} f(``at*x\^{}2'',3,5) // computes 4*3\^{}2 not 5*3\^{}2

\begin{verbatim}
36
\end{verbatim}

Jika Anda ingin menggunakan nilai lain untuk ``at'' selain nilai global, Anda perlu menambahkan ``at=value''.

\textgreater at:=4; function f(expr,x,a) := expr(x,at=a); \ldots{}

\textgreater{} f(``at*x\^{}2'',3,5)

\begin{verbatim}
45
\end{verbatim}

Sebagai referensi, kami menyatakan bahwa koleksi panggilan (dibahas di tempat lain) dapat berisi ekspresi. Jadi kita dapat membuat contoh di atas sebagai berikut.

\textgreater at:=4; function f(expr,x) := expr(x); \ldots{}

\textgreater{} f(\{\{``at*x\^{}2'',at=5\}\},3)

\begin{verbatim}
45
\end{verbatim}

Ekspresi dalam x sering digunakan seperti halnya fungsi.

Perhatikan bahwa mendefinisikan fungsi dengan nama yang sama seperti ekspresi simbolik global akan menghapus variabel ini untuk menghindari kebingungan antara ekspresi simbolik dan fungsi.

\textgreater f \&= 5*x;

\textgreater function f(x) := 6*x;

\textgreater f(2)

\begin{verbatim}
12
\end{verbatim}

Sesuai dengan konvensi, ekspresi simbolik atau numerik harus diberi nama fx, fxy, dll. Skema penamaan ini tidak boleh digunakan untuk fungsi.

\textgreater fx \&= diff(x\^{}x,x); \(&fx\)\(x^{x}\,\left(\log x+1\right)\)\$Bentuk khusus dari sebuah ekspresi memungkinkan variabel apa pun sebagai parameter tanpa nama untuk evaluasi ekspresi, bukan hanya ``x'', ``y'', dll. Untuk ini, mulailah ekspresi dengan ``@(variabel)\ldots{}''.

\textgreater{}``@(a,b) a\textsuperscript{2+b}2'', \%(4,5)

\begin{verbatim}
@(a,b) a^2+b^2
41
\end{verbatim}

Hal ini memungkinkan untuk memanipulasi ekspresi dalam variabel lain untuk fungsi EMT yang membutuhkan ekspresi dalam ``x''.

Cara paling dasar untuk mendefinisikan fungsi sederhana adalah dengan menyimpan rumusnya dalam ekspresi simbolik atau numerik. Jika variabel utamanya adalah x, ekspresi tersebut dapat dievaluasi seperti halnya sebuah fungsi.

Seperti yang Anda lihat pada contoh berikut, variabel global terlihat selama evaluasi

\textgreater fx \&= x\^{}3-a*x; \ldots{}\\
\textgreater{} a=1.2; fx(0.5)

\begin{verbatim}
-0.475
\end{verbatim}

Semua variabel lain dalam ekspresi dapat ditentukan dalam evaluasi menggunakan parameter yang ditetapkan.

\textgreater fx(0.5,a=1.1)

\begin{verbatim}
-0.425
\end{verbatim}

Sebuah ekspresi tidak perlu simbolis. Ini diperlukan, jika ekspresi berisi fungsi, yang hanya diketahui di kernel numerik, bukan di Maxima.

\chapter{Symbolic Mathematics}\label{symbolic-mathematics}

EMT melakukan matematika simbolik dengan bantuan Maxima. Untuk detailnya, mulailah dengan tutorial berikut ini, atau telusuri referensi untuk Maxima. Para ahli dalam Maxima harus mencatat bahwa ada perbedaan dalam sintaks antara sintaks asli Maxima dan sintaks default ekspresi simbolik dalam EMT.

Matematika simbolik diintegrasikan dengan mulus ke dalam Euler dengan \&. Setiap ekspresi yang dimulai dengan \& adalah ekspresi simbolik. Ekspresi ini dievaluasi dan dicetak oleh Maxima.

Pertama-tama, Maxima memiliki aritmatika ``tak terbatas'' yang dapat menangani angka yang sangat besar.

\textgreater{}\(&44!\)\(2658271574788448768043625811014615890319638528000000000\)\(Dengan cara ini, kita dapat menghitung hasil yang besar dengan tepat.
Mari kita hitung !\)\(C(44,10) = \frac{44!}{34! \cdot 10!}\)\(\>\)\& 44!/(34!*10!) // nilai C(44,10) \[2481256778\]Tentu saja, Maxima memiliki fungsi yang lebih efisien untuk hal ini (seperti halnya bagian numerik EMT).

\textgreater{}\(binomial(44,10) //menghitung C(44,10) menggunakan fungsi binomial()\)\(2481256778\)\$Untuk mempelajari lebih lanjut tentang fungsi tertentu klik dua kali di atasnya. Misalnya, coba klik dua kali pada ``\&binomial'' di baris perintah sebelumnya. Ini membuka dokumentasi Maxima seperti yang disediakan oleh penulis program itu.

Anda akan belajar bahwa yang berikut ini juga berfungsi. \[C(x,3)=\frac{x!}{(x-3)!3!}=\frac{(x-2)(x-1)x}{6}\]\textgreater{}\(binomial(x,3) // C(x,3)\)\(\frac{\left(x-2\right)\,\left(x-1\right)\,x}{6}\)\$Jika Anda ingin mengganti x dengan nilai tertentu, gunakan ``with''.

\textgreater{}\(&binomial(x,3) with x=10 // substitusi x=10 ke C(x,3)\)\(120\)\$Dengan begitu, Anda dapat menggunakan solusi dari sebuah persamaan dalam persamaan lain.

Ekspresi simbolik dicetak oleh Maxima dalam bentuk 2D. Alasannya adalah sebuah bendera simbolik khusus dalam string. Seperti yang telah Anda lihat pada contoh sebelumnya dan contoh berikut, jika Anda telah menginstal LaTeX, Anda dapat mencetak ekspresi simbolik dengan Latex. Jika tidak, perintah berikut ini akan mengeluarkan pesan kesalahan. Untuk mencetak ekspresi simbolik dengan LaTeX, gunakan \$ di depan \& (atau Anda dapat menghilangkan \&) sebelum perintah. Jangan jalankan perintah Maxima dengan \$, jika Anda tidak memiliki LaTeX.

\textgreater{}\((3+x)/(x^2+1)\)\(\frac{x+3}{x^2+1}\)\$Ekspresi simbolik diuraikan oleh Euler. Jika Anda membutuhkan sintaks yang kompleks dalam satu ekspresi, Anda dapat mengapit ekspresi dalam ``\ldots{}''. Menggunakan lebih dari satu ekspresi sederhana dimungkinkan, tetapi sangat tidak disarankan.

\textgreater\&``v := 5; v\^{}2''

\begin{verbatim}
                                  25    
\end{verbatim}

Untuk kelengkapan, kami menyatakan bahwa ekspresi simbolik dapat digunakan dalam program, tetapi harus diapit dengan tanda kutip. Selain itu, akan jauh lebih efektif untuk memanggil Maxima pada saat kompilasi jika memungkinkan.

\textgreater\$\&expand((1+x)\^{}4), \(&factor(diff(%,x)) // diff: turunan, factor: faktor
\)\(4\,\left(x+1\right)^3\)\$\pandocbounded{\includegraphics[keepaspectratio]{images/EMT untuk Perhitungan Aljabar_Isni Azizah Utami_23030630016-017.png}}

Sekali lagi, \% mengacu pada hasil sebelumnya.

Untuk mempermudah, kita menyimpan solusi ke dalam sebuah variabel simbolik. Variabel simbolik didefinisikan dengan ``\&=''.

\textgreater fx \&= (x+1)/(x\^{}4+1); \(&fx\)\(\frac{x+1}{x^4+1}\)\$Ekspresi simbolik dapat digunakan dalam ekspresi simbolik lainnya.

\textgreater{}\(&factor(diff(fx,x))\)\(\frac{-3\,x^4-4\,x^3+1}{\left(x^4+1\right)^2}\)\$Masukan langsung dari perintah Maxima juga tersedia. Mulai baris perintah dengan ``::''. Sintaks Maxima disesuaikan dengan sintaks EMT (disebut ``mode kompatibilitas'').

\textgreater\&factor(20!)

\begin{verbatim}
                         2432902008176640000
\end{verbatim}

\textgreater::: factor(10!)

\begin{verbatim}
                               8  4  2
                              2  3  5  7
\end{verbatim}

\textgreater:: factor(20!)

\begin{verbatim}
                        18  8  4  2
                       2   3  5  7  11 13 17 19
\end{verbatim}

Jika Anda adalah seorang ahli dalam Maxima, Anda mungkin ingin menggunakan sintaks asli Maxima. Anda dapat melakukan ini dengan ``:::''.

\textgreater::: av:g\$ av\^{}2;

\begin{verbatim}
                                   2
                                  g
\end{verbatim}

\textgreater fx \&= x\^{}3*exp(x), \$fx

\begin{verbatim}
                                 3  x
                                x  E
\end{verbatim}

\[x^3\,e^{x}\]Variabel tersebut dapat digunakan dalam ekspresi simbolik lainnya. Perhatikan, bahwa pada perintah berikut ini, sisi kanan dari \&= dievaluasi sebelum penugasan ke Fx.

\textgreater\&(fx with x=5), \$\%, \&float(\%)

\begin{verbatim}
                                     5
                                125 E
\end{verbatim}

\[125\,e^5\] 18551.64488782208

\textgreater fx(5)

\begin{verbatim}
18551.6448878
\end{verbatim}

Untuk mengevaluasi ekspresi dengan nilai variabel tertentu, Anda dapat menggunakan operator ``with''. Baris perintah berikut ini juga mendemonstrasikan bahwa Maxima dapat mengevaluasi sebuah ekspresi secara numerik dengan float().

\textgreater\&(fx with x=10)-(fx with x=5), \&float(\%)

\begin{verbatim}
                                10        5
                          1000 E   - 125 E


                         2.20079141499189e+7
\end{verbatim}

\textgreater{}\(factor(diff(fx,x,2))\)\(x\,\left(x^2+6\,x+6\right)\,e^{x}\)\$Untuk mendapatkan kode Latex untuk sebuah ekspresi, Anda dapat menggunakan perintah tex.

\textgreater tex(fx)

\begin{verbatim}
x^3\,e^{x}
\end{verbatim}

Ekspresi simbolik dapat dievaluasi seperti halnya ekspresi numerik.

\textgreater fx(0.5)

\begin{verbatim}
0.206090158838
\end{verbatim}

Dalam ekspresi simbolik, hal ini tidak dapat dilakukan, karena Maxima tidak mendukungnya. Sebagai gantinya, gunakan sintaks ``with'' (bentuk yang lebih baik dari perintah at(\ldots) pada Maxima).

\textgreater{}\(&fx with x=1/2\)\(\frac{\sqrt{e}}{8}\)\$Penugasan ini juga bisa bersifat simbolis.

\textgreater\$\&fx with x=1+t

\[\left(t+1\right)^3\,e^{t+1}\]Perintah ``solve'' menyelesaikan ekspresi simbolik untuk sebuah variabel di Maxima. Hasilnya adalah sebuah vektor solusi.

\textgreater\$\&solve(x\^{}2+x=4,x)

\$\left[ x=\frac{-\sqrt{17}-1}{2} , x=\frac{\sqrt{17}-1}{2} \right] \$\$

Bandingkan dengan perintah ``solve'' numerik di Euler, yang membutuhkan nilai awal, dan secara opsional nilai target.

\textgreater solve(``x\^{}2+x'',1,y=4)

\begin{verbatim}
1.56155281281
\end{verbatim}

Nilai numerik dari solusi simbolik dapat dihitung dengan evaluasi hasil simbolik. Euler akan membaca penugasan x= dst. Jika Anda tidak membutuhkan hasil numerik untuk perhitungan lebih lanjut, Anda juga bisa membiarkan Maxima menemukan nilai numeriknya

\textgreater sol \&= solve(x\^{}2+2*x=4,x); \$\&sol, sol(), \$\&float(sol)

\$\left[ x=-\sqrt{5}-1 , x=\sqrt{5}-1 \right] \$\$ {[}-3.23607, 1.23607{]}

\$\left[ x=-3.23606797749979 , x=1.23606797749979 \right] \$\$

Untuk mendapatkan solusi simbolik yang spesifik, seseorang dapat menggunakan ``with'' dan indeks.

\textgreater\$\&solve(x\^{}2+x=1,x), x2 \&= x with \%{[}2{]}; \(&x2\)\(\frac{\sqrt{5}-1}{2}\)\$\pandocbounded{\includegraphics[keepaspectratio]{images/EMT untuk Perhitungan Aljabar_Isni Azizah Utami_23030630016-029.png}}

Untuk menyelesaikan sistem persamaan, gunakan vektor persamaan. Hasilnya adalah vektor solusi.

\textgreater sol \&= solve({[}x+y=3,x\textsuperscript{2+y}2=5{]},{[}x,y{]}); \$\&sol, \(&x\*y with sol[1]\)\(2\)\$\pandocbounded{\includegraphics[keepaspectratio]{images/EMT untuk Perhitungan Aljabar_Isni Azizah Utami_23030630016-031.png}}

Ekspresi simbolik dapat memiliki flags, yang menunjukkan perlakuan khusus di Maxima. Beberapa flag dapat digunakan sebagai perintah juga, namun ada juga yang tidak. Flags ditambahkan dengan ``\textbar{}'' (bentuk yang lebih baik dari ``ev(\ldots,flags)'')

\textgreater{}\(& diff((x^3-1)/(x+1),x) //turunan bentuk pecahan\)\(\frac{3\,x^2}{x+1}-\frac{x^3-1}{\left(x+1\right)^2}\)\(\>\)\& diff((x\^{}3-1)/(x+1),x) \textbar{} ratsimp //menyederhanakan pecahan \[\frac{2\,x^3+3\,x^2+1}{x^2+2\,x+1}\]\textgreater{}\(&factor(%)
\)\(\frac{2\,x^3+3\,x^2+1}{\left(x+1\right)^2}\)\$

\chapter{Fungsi}\label{fungsi}

Di EMT fungsi adalah program yang didefenisikan dengan perintah ``function''. Fungsi dapat menjadi fungsi satu baris atau fungsi multi baris. Fungsi satu baris dapat berupa numerik atau simbolis. Fungsi satu baris didefinisikan oleh ``:=''.

\textgreater function f(x) := x*sqrt(x\^{}2+1)

Sebagai gambaran umum, kami menunjukkan semua definisi yang mungkin untuk fungsi satu baris. Sebuah fungsi dapat dievaluasi seperti halnya fungsi Euler bawaan.

\textgreater f(2)

\begin{verbatim}
4.472135955
\end{verbatim}

Fungsi ini dapat digunakan juga dalam vektor, dengan mengikuti aturan bahasa matrik Euler, karena ekspresi yang digunakan dalam fungsi divektorkan.Kita akan mencobanya menggunakan fungsi f di atas.

\textgreater f(0:0.1:1)

\begin{verbatim}
[0,  0.100499,  0.203961,  0.313209,  0.430813,  0.559017,  0.699714,
0.854459,  1.0245,  1.21083,  1.41421]
\end{verbatim}

Fungsi juga dapat menjadi plot, hanya dengan memberikan nama fungsi. Berbeda dengan ekpresi simbolik atau numerik, nama fungsi harus diberikan dalam string.

\textgreater solve(``f'',1,y=1)

\begin{verbatim}
0.786151377757
\end{verbatim}

Secara default, jika Anda perlu menimpa fungsi built-in, Anda harus menambahkan kata kunci ``overwrite''. Menimpa fungsi bawaan berbahaya dan dapat menyebabkan masalah bagi fungsi lain yang bergantung pada fungsi tersebut. Anda masih dapat memanggil fungsi bawaan sebagai ``\_\ldots'', jika fungsi tersebut merupakan fungsi dalam inti Euler.

\textgreater function overwrite sin (x) := \_sin(x°) // redine sine in degrees

\textgreater sin(45)

\begin{verbatim}
0.707106781187
\end{verbatim}

Jika ingin menghapus definisi dari sin dan mendefinisikannya ulang, menggunakan perintah ``forget''

\textgreater forget sin; sin(pi/4)

\begin{verbatim}
0.707106781187
\end{verbatim}

\chapter{Default Parameters}\label{default-parameters}

Parameter default adalah fungsi parameter yang memiliki nilai awal. Fungsi numerik dapat memiliki parameter default.

\textgreater function f(x,a=1) := a*x\^{}2

Menghilangkan parameter ini menggunakan nilai default.

\textgreater f(4)

\begin{verbatim}
16
\end{verbatim}

Menetapkannya akan menimpa nilai default.

\textgreater f(4,5)

\begin{verbatim}
80
\end{verbatim}

Parameter yang ditetapkan juga menimpanya. Ini digunakan oleh banyak fungsi Euler seperti plot2d, plot3d.

\textgreater f(4,a=1)

\begin{verbatim}
16
\end{verbatim}

Jika sebuah variabel bukan parameter, maka variabel tersebut harus bersifat global. Fungsi satu baris dapat melihat variabel global.

\textgreater function f(x) := a*x\^{}2 \textgreater a=6; f(2)

\begin{verbatim}
24
\end{verbatim}

Tetapi parameter yang ditetapkan akan menggantikan nilai global. Jika argumen tidak ada dalam daftar parameter yang telah ditetapkan sebelumnya, argumen tersebut harus dideklarasikan dengan ``:=''!

\textgreater f(2,a:=5)

\begin{verbatim}
20
\end{verbatim}

Fungsi simbolik didefinisikan dengan ``\&=''. Fungsi-fungsi ini didefinisikan dalam Euler dan Maxima, dan dapat digunakan di kedua bahasa tersebut. Ekspresi pendefinisian dijalankan melalui Maxima sebelum definisi.

\textgreater function g(x) \&= x\^{}3-x*exp(-x); \(&g(x)\)\(x^3-x\,e^ {- x }\)\$Fungsi simbolik dapat digunakan dalam ekspresi simbolik

\textgreater\$\&diff(g(x),x), \(&% with x=4/3
\)\(\frac{e^ {- \frac{4}{3} }}{3}+\frac{16}{3}\)\$\pandocbounded{\includegraphics[keepaspectratio]{images/EMT untuk Perhitungan Aljabar_Isni Azizah Utami_23030630016-037.png}}

Itu juga dapat digunakan dalam ekspresi numerik. Tentu saja, ini hanya akan berfungsi jika EMT dapat mengintrepertasikan semua yang ada di dalam fungsi tersebut.

\textgreater g(5+g(1))

\begin{verbatim}
178.635099908
\end{verbatim}

Itu juga dapat digunakan untuk mendefinisikan fungsi atau ekspresi simbolik lainnya.

\textgreater function G(x) \&= factor(integrate(g(x),x)); \(&G(c) // integrate: mengintegralkan\)\(\frac{e^ {- c }\,\left(c^4\,e^{c}+4\,c+4\right)}{4}\)\$\textgreater solve(\&g(x),0.5)

\begin{verbatim}
0.703467422498
\end{verbatim}

Berikut ini juga berfungsi, karena Euler menggunakan ekspresi simbolis dalam fungsi g, jika tidak menemukan variabel simbolik g, dan jika ada fungsi simbolis g.

\textgreater solve(\&g,0.5)

\begin{verbatim}
0.703467422498
\end{verbatim}

\textgreater function P(x,n) \&= (2*x-1)\^{}n; \(&P(x,n)\)\(\left(2\,x-1\right)^{n}\)\$\textgreater function Q(x,n) \&= (x+2)\^{}n; \(&Q(x,n)\)\(\left(x+2\right)^{n}\)\(\>\)\&P(x,4), \(&expand(%)
\)\(16\,x^4-32\,x^3+24\,x^2-8\,x+1\)\$\pandocbounded{\includegraphics[keepaspectratio]{images/EMT untuk Perhitungan Aljabar_Isni Azizah Utami_23030630016-042.png}}

\textgreater P(3,4)

\begin{verbatim}
625
\end{verbatim}

\textgreater\$\&P(x,4)+ Q(x,3), \(&expand(%)
\)\(16\,x^4-31\,x^3+30\,x^2+4\,x+9\)\$\pandocbounded{\includegraphics[keepaspectratio]{images/EMT untuk Perhitungan Aljabar_Isni Azizah Utami_23030630016-044.png}}

\textgreater\$\&P(x,4)-Q(x,3), \$\&expand(\%), \(&factor(%)
\)\(16\,x^4-33\,x^3+18\,x^2-20\,x-7\)\$\pandocbounded{\includegraphics[keepaspectratio]{images/EMT untuk Perhitungan Aljabar_Isni Azizah Utami_23030630016-046.png}}

\begin{figure}
\centering
\pandocbounded{\includegraphics[keepaspectratio]{images/EMT untuk Perhitungan Aljabar_Isni Azizah Utami_23030630016-047.png}}
\caption{images/EMT\%20untuk\%20Perhitungan\%20Aljabar\_Isni\%20Azizah\%20Utami\_23030630016-047.png}
\end{figure}

\textgreater\$\&P(x,4)*Q(x,3), \$\&expand(\%), \(&factor(%)
\)\(\left(x+2\right)^3\,\left(2\,x-1\right)^4\)\$\pandocbounded{\includegraphics[keepaspectratio]{images/EMT untuk Perhitungan Aljabar_Isni Azizah Utami_23030630016-049.png}}

\begin{figure}
\centering
\pandocbounded{\includegraphics[keepaspectratio]{images/EMT untuk Perhitungan Aljabar_Isni Azizah Utami_23030630016-050.png}}
\caption{images/EMT\%20untuk\%20Perhitungan\%20Aljabar\_Isni\%20Azizah\%20Utami\_23030630016-050.png}
\end{figure}

\textgreater\$\&P(x,4)/Q(x,1), \$\&expand(\%), \(&factor(%)
\)\(\frac{\left(2\,x-1\right)^4}{x+2}\)\$\pandocbounded{\includegraphics[keepaspectratio]{images/EMT untuk Perhitungan Aljabar_Isni Azizah Utami_23030630016-052.png}}

\begin{figure}
\centering
\pandocbounded{\includegraphics[keepaspectratio]{images/EMT untuk Perhitungan Aljabar_Isni Azizah Utami_23030630016-053.png}}
\caption{images/EMT\%20untuk\%20Perhitungan\%20Aljabar\_Isni\%20Azizah\%20Utami\_23030630016-053.png}
\end{figure}

\textgreater function f(x) \&= x\^{}3-x; \(&f(x)\)\(x^3-x\)\$Dengan \&= fungsinya simbolis, dan dapat digunakan dalam ekspresi simbolik lainnya. Contohnya dalam integral tak tentu sebagai berikut.

\textgreater{}\(&integrate(f(x),x)\)\(\frac{x^4}{4}-\frac{x^2}{2}\)\(Dengan := fungsinya numerik. Contoh yang baik adalah integral tak tentu seperti\)\(f(x) = \int_1^x t^t \, dt,\)\$ yang tidak dapat dievaluasi secara simbolik.

Jika kita mendefinisikan ulang fungsi tersebut dengan kata kunci ``map'', maka fungsi ini dapat digunakan untuk vektor x. Secara internal, fungsi ini dipanggil untuk semua nilai x satu kali, dan hasilnya disimpan dalam sebuah vektor.

\textgreater function map f(x) := integrate(``x\^{}x'',1,x) \textgreater f(0:0.5:2)

\begin{verbatim}
[-0.783431,  -0.410816,  0,  0.676863,  2.05045]
\end{verbatim}

Fungsi dapat memiliki nilai default untuk parameter.

\textgreater function mylog (x,base=10) := ln(x)/ln(base);

Sekarang fungsi dapat dipanggil dengan menggunakan suatau parameter ``base'' maupun tidak.

\textgreater mylog(100), mylog(2\^{}6.7,2)

\begin{verbatim}
2
6.7
\end{verbatim}

Selain itu, dimungkinkan untuk menggunakan parameter yang ditetapkan.

\textgreater mylog(E\^{}2,base=E)

\begin{verbatim}
2
\end{verbatim}

Sering kali, kita ingin menggunakan fungsi untuk vektor di satu tempat, dan untuk elemen individual di tempat lain. Ini tapat terjadi dengan vektor parameter.

\textgreater function f({[}a,b{]}) \&= a\textsuperscript{2+b}2-a*b+b; \$\&f(a,b), \(&f(x,y)\)\(y^2-x\,y+y+x^2\)\$\pandocbounded{\includegraphics[keepaspectratio]{images/EMT untuk Perhitungan Aljabar_Isni Azizah Utami_23030630016-058.png}}

Fungsi simbolik seperti itu dapat digunakan untuk variabel simbolik. Tetapi fungsi ini juga dapat digunakan untuk vektor numerik.

\textgreater v={[}3,4{]}; f(v)

\begin{verbatim}
17
\end{verbatim}

Ada juga fungsi yang murni simbolis, yang tidak dapat digunakan secara numerik.

\textgreater function lapl(expr,x,y) \&\&= diff(expr,x,2)+diff(expr,y,2)//turunan parsial kedua

\begin{verbatim}
                 diff(expr, y, 2) + diff(expr, x, 2)
\end{verbatim}

\textgreater\$\&realpart((x+I*y)\^{}4), \(&lapl(%,x,y)
\)\(0\)\$\pandocbounded{\includegraphics[keepaspectratio]{images/EMT untuk Perhitungan Aljabar_Isni Azizah Utami_23030630016-060.png}}

Tetapi tentu saja, mereka dapat digunakan dalam ekspresi simbolis atau dalam definisi fungsi simbolis.

\textgreater function f(x,y) \&= factor(lapl((x+y\textsuperscript{2)}5,x,y)); \(&f(x,y)\)\(10\,\left(y^2+x\right)^3\,\left(9\,y^2+x+2\right)\)\$Ringkasan: 1. \&= mendefinisikan fungsi simbolis, 2. := mendefinisikan fungsi numerik, 3. \&\&= mendefinisikan fungsi simbolis murni

\chapter{Memecahkan Ekspresi}\label{memecahkan-ekspresi}

Ekspresi dapat diselesaikan secara numerik dan simbolis. Untuk menyelesaikan ekspresi sederhana dari satu variabel, kita dapat menggunakan fungsi solve(). Perlu nilai awal untuk memulai pencarian. Secara internal, solve() menggunakan metode secant.

\textgreater solve(``x\^{}2-2'',1)

\begin{verbatim}
1.41421356237
\end{verbatim}

Ini juga berfungsi untuk fungsi simbolik, perhatikan fungsi berikut ini.

\textgreater\$\&solve(x\^{}2=2,x)

\$\left[ x=-\sqrt{2} , x=\sqrt{2} \right] \$\$

\textgreater\$\&solve(x\^{}2-2,x)

\$\left[ x=-\sqrt{2} , x=\sqrt{2} \right] \$\$

\textgreater\$\&solve(a*x\^{}2+b*x+c=0,x)

\$\left[ x=\frac{-\sqrt{b^2-4\,a\,c}-b}{2\,a} , x=\frac{\sqrt{b^2-4\,  a\,c}-b}{2\,a} \right] \$\$

\textgreater\$\&solve({[}a*x+b*y=c,d*x+e*y=f{]},{[}x,y{]})

\$\left[ \left[ x=-\frac{c\,e}{b\,\left(d-5\right)-a\,e} , y=\frac{c  \,\left(d-5\right)}{b\,\left(d-5\right)-a\,e} \right]  \right{]} \$\$

\textgreater px \&= 4*x\textsuperscript{8+x}7-x\^{}4-x; \(&px\)\(4\,x^8+x^7-x^4-x\)\$Sekarang kita mencari titik, di mana polinomialnya adalah 2. Dalam solve(), nilai target default y=0 dapat diubah dengan variabel yang ditetapkan.

Kami menggunakan y=2 dan memeriksa dengan mengevaluasi polinomial pada hasil sebelumnya.

\textgreater solve(px,1,y=2), px(\%)

\begin{verbatim}
0.966715594851
2
\end{verbatim}

Memecahkan ekspresi simbolis dalam bentuk simbolis mengembalikan daftar solusi. Kami menggunakan pemecah simbolik solve() yang disediakan oleh Maxima.

\textgreater sol \&= solve(x\^{}2-x-1,x); \$\&sol

\$\left[ x=\frac{1-\sqrt{5}}{2} , x=\frac{\sqrt{5}+1}{2} \right] \$\$

Cara termudah untuk mendapatkan nilai numerik adalah dengan mengevaluasi solusi secara numerik seperti ekspresi.

\textgreater longest sol()

\begin{verbatim}
    -0.6180339887498949       1.618033988749895 
\end{verbatim}

Untuk menggunakan solusi secara simbolis dalam ekspresi lain, cara termudah adalah ``with''.

\textgreater\$\&x\^{}2 with sol{[}1{]}, \(&expand(x^2-x-1 with sol[2])\)\(0\)\$\pandocbounded{\includegraphics[keepaspectratio]{images/EMT untuk Perhitungan Aljabar_Isni Azizah Utami_23030630016-069.png}}

Memecahkan sistem persamaan secara simbolis dapat dilakukan dengan vektor persamaan dan solver simbolis solve(). Hasilnya dalam bentuk persamaan.

\textgreater\$\&solve({[}x+y=2,x\^{}3+2*y+x=4{]},{[}x,y{]})

\$\left[ \left[ x=-1 , y=3 \right]  , \left[ x=1 , y=1 \right]  , \left[ x=0 , y=2 \right]  \right{]} \$\$

Fungsi f() dapat melihat variabel global. Namun seringkali kita ingin menggunakan parameter lokal. \[a^x-x^a = 0.1\]dengan a=3. \textgreater function f(x,a) := x\textsuperscript{a-a}x;

Salah satu cara untuk mengoper parameter tambahan ke f() adalah dengan menggunakan sebuah daftar dengan nama fungsi dan parameternya (cara-cara lainnya adalah parameter titik koma).

\textgreater solve(\{\{``f'',3\}\},2,y=0.1)

\begin{verbatim}
2.54116291558
\end{verbatim}

Ini juga bekerja dengan ekspresi. Tapi daftar elemen yang ada harus digunakan. (Lebih lanjut tentang daftar dalam tutorial tentang sintaks EMT).

\textgreater solve(\{\{``x\textsuperscript{a-a}x'',a=3\}\},2,y=0.1)

\begin{verbatim}
2.54116291558
\end{verbatim}

\chapter{Menyelesaikan Pertidaksamaan}\label{menyelesaikan-pertidaksamaan}

Untuk menyelesaikan pertidaksamaan, EMT tidak akan dapat melakukannya, melainkan dengan bantuan Maxima, artinya secara eksak (simbolik). Perintah Maxima yang digunakan adalah fourier\_elim(), yang harus dipanggil dengan perintah ``load(fourier\_elim)'' terlebih dahulu.

\textgreater\&load(fourier\_elim)

\begin{verbatim}
        D:/Euler x64/maxima/share/maxima/5.35.1/share/fourier_elim/fo\
urier_elim.lisp
\end{verbatim}

\textgreater\$\&fourier\_elim({[}x\^{}2 - 1\textgreater0{]},{[}x{]}) // x\^{}2-1 \textgreater{} 0

\$\left[ 1<x \right] \lor \left[ x<-1 \right] \$\$

\textgreater\$\&fourier\_elim({[}x\^{}2 - 1\textless0{]},{[}x{]}) // x\^{}2-1 \textless{} 0

\$\left[ -1<x , x<1 \right] \$\$

\textgreater\$\&fourier\_elim({[}x\^{}2 - 1 \# 0{]},{[}x{]}) // x\^{}-1 \textless\textgreater{} 0

\$\left[ -1<x , x<1 \right] \lor \left[ 1<x \right] \lor \left[ x<-1   \right] \$\$

\textgreater\$\&fourier\_elim({[}x \# 6{]},{[}x{]})

\$\left[ x<6 \right] \lor \left[ 6<x \right] \$\$

\textgreater{}\(&fourier\_elim([x < 1, x \> 1],[x]) // tidak memiliki penyelesaian\)\({\it emptyset}\)\(\>\)\&fourier\_elim({[}minf \textless{} x, x \textless{} inf{]},{[}x{]}) // solusinya R \[{\it universalset}\]\textgreater\$\&fourier\_elim({[}x\^{}3 - 1 \textgreater{} 0{]},{[}x{]})

\$\left[ 1<x , x^2+x+1>0 \right] \lor \left[ x<1 , -x^2-x-1>0   \right] \$\$

\textgreater\$\&fourier\_elim({[}cos(x) \textless{} 1/2{]},{[}x{]}) // ??? gagal

\$\left[ 1-2\,\cos x>0 \right] \$\$

\textgreater\$\&fourier\_elim({[}y-x \textless{} 5, x - y \textless{} 7, 10 \textless{} y{]},{[}x,y{]}) // sistem pertidaksamaan

\$\left[ y-5<x , x<y+7 , 10<y \right] \$\$

\textgreater\$\&fourier\_elim({[}y-x \textless{} 5, x - y \textless{} 7, 10 \textless{} y{]},{[}y,x{]})

\$\left[ {\it max}\left(10 , x-7\right)<y , y<x+5 , 5<x \right] \$\$

\textgreater\$\&fourier\_elim((x + y \textless{} 5) and (x - y \textgreater8),{[}x,y{]})

\$\left[ y+8<x , x<5-y , y<-\frac{3}{2} \right] \$\$

\textgreater\$\&fourier\_elim(((x + y \textless{} 5) and x \textless{} 1) or (x - y \textgreater8),{[}x,y{]})

\$\left[ y+8<x \right] \lor \left[ x<{\it min}\left(1 , 5-y\right)   \right] \$\$

\textgreater\&fourier\_elim({[}max(x,y) \textgreater{} 6, x \# 8, abs(y-1) \textgreater{} 12{]},{[}x,y{]})

\begin{verbatim}
        [6 &lt; x, x &lt; 8, y &lt; - 11] or [8 &lt; x, y &lt; - 11]
 or [x &lt; 8, 13 &lt; y] or [x = y, 13 &lt; y] or [8 &lt; x, x &lt; y, 13 &lt; y]
 or [y &lt; x, 13 &lt; y]    
\end{verbatim}

\textgreater\$\&fourier\_elim({[}(x+6)/(x-9) \textless= 6{]},{[}x{]})

\$\left[ x=12 \right] \lor \left[ 12<x \right] \lor \left[ x<9   \right] \$\$

\chapter{Bahasa Matriks}\label{bahasa-matriks}

Dokumentasi inti EMT berisi diskusi terperinci tentang bahasa matriks Euler. Vektor dan matriks dimasukkan dengan tanda kurung siku, elemen dipisahkan dengan koma, baris dipisahkan dengan titik koma.

\textgreater A={[}1,2;3,4{]}

\begin{verbatim}
            1             2 
            3             4 
\end{verbatim}

Hasil kali matriks dilambangkan dengan sebuah titik.

\textgreater b={[}3;4{]}

\begin{verbatim}
            3 
            4 
\end{verbatim}

\textgreater b' // transpose b

\begin{verbatim}
[3,  4]
\end{verbatim}

\textgreater inv(A) //inverse A

\begin{verbatim}
           -2             1 
          1.5          -0.5 
\end{verbatim}

\textgreater A.b //perkalian matriks

\begin{verbatim}
           11 
           25 
\end{verbatim}

\textgreater A.inv(A)

\begin{verbatim}
            1             0 
            0             1 
\end{verbatim}

Poin utama dari bahasa matriks adalah bahwa semua fungsi dan operator bekerja elemen demi elemen.

\textgreater A.A

\begin{verbatim}
            7            10 
           15            22 
\end{verbatim}

\textgreater A\^{}2 //perpangkatan elemen2 A

\begin{verbatim}
            1             4 
            9            16 
\end{verbatim}

\textgreater A.A.A

\begin{verbatim}
           37            54 
           81           118 
\end{verbatim}

\textgreater power(A,3) //perpangkatan matriks

\begin{verbatim}
           37            54 
           81           118 
\end{verbatim}

\textgreater A/A //pembagian elemen-elemen matriks yang seletak

\begin{verbatim}
            1             1 
            1             1 
\end{verbatim}

\textgreater A/b //pembagian elemen2 A oleh elemen2 b kolom demi kolom (karena b vektor kolom)

\begin{verbatim}
     0.333333      0.666667 
         0.75             1 
\end{verbatim}

\textgreater A\textbackslash b // hasilkali invers A dan b, A\^{}(-1)b

\begin{verbatim}
           -2 
          2.5 
\end{verbatim}

\textgreater inv(A).b

\begin{verbatim}
           -2 
          2.5 
\end{verbatim}

\textgreater A\textbackslash A //A\^{}(-1)A

\begin{verbatim}
            1             0 
            0             1 
\end{verbatim}

\textgreater inv(A).A

\begin{verbatim}
            1             0 
            0             1 
\end{verbatim}

\textgreater A*A //perkalin elemen-elemen matriks seletak

\begin{verbatim}
            1             4 
            9            16 
\end{verbatim}

Ini bukan hasil kali matriks, tetapi perkalian elemen demi elemen. Hal yang sama berlaku untuk vektor.

\textgreater b\^{}2 // perpangkatan elemen-elemen matriks/vektor

\begin{verbatim}
            9 
           16 
\end{verbatim}

Jika salah satu operan adalah vektor atau skalar, maka operan tersebut akan diperluas dengan cara alami.

\textgreater2*A

\begin{verbatim}
            2             4 
            6             8 
\end{verbatim}

Misalnya, jika operan adalah vektor kolom, elemen-elemennya diterapkan ke semua baris A.

\textgreater{[}1,2{]}*A

\begin{verbatim}
            1             4 
            3             8 
\end{verbatim}

Jika operan adalah vektor baris, elemen-elemennya diterapkan ke semua kolom A.

\textgreater A*{[}2,3{]}

\begin{verbatim}
            2             6 
            6            12 
\end{verbatim}

Kita dapat membayangkan perkalian ini seolah-olah vektor baris v telah diduplikasi untuk membentuk matriks dengan ukuran yang sama dengan A.

\textgreater dup({[}1,2{]},2) // dup: menduplikasi/menggandakan vektor {[}1,2{]} sebanyak 2 kali (baris)

\begin{verbatim}
            1             2 
            1             2 
\end{verbatim}

\textgreater A*dup({[}1,2{]},2)

\begin{verbatim}
            1             4 
            3             8 
\end{verbatim}

Hal ini juga berlaku untuk dua vektor di mana satu vektor adalah vektor baris dan yang lainnya adalah vektor kolom. Kami menghitung i*j untuk i, j dari 1 sampai 5. Caranya adalah dengan mengalikan 1:5 dengan transposenya. Bahasa matriks Euler secara otomatis menghasilkan sebuah tabel nilai.

\textgreater(1:5)*(1:5)' // hasilkali elemen-elemen vektor baris dan vektor kolom

\begin{verbatim}
            1             2             3             4             5 
            2             4             6             8            10 
            3             6             9            12            15 
            4             8            12            16            20 
            5            10            15            20            25 
\end{verbatim}

Sekali lagi, ingatlah bahwa ini bukan produk matriks!

\textgreater(1:5).(1:5)' // hasilkali vektor baris dan vektor kolom

\begin{verbatim}
55
\end{verbatim}

\textgreater sum((1:5)*(1:5)) // sama hasilnya

\begin{verbatim}
55
\end{verbatim}

Bahkan operator seperti \textless{} atau == bekerja dengan cara yang sama.

\textgreater(1:10)\textless6 // menguji elemen-elemen yang kurang dari 6

\begin{verbatim}
[1,  1,  1,  1,  1,  0,  0,  0,  0,  0]
\end{verbatim}

Sebagai contoh, kita dapat menghitung jumlah elemen yang memenuhi kondisi tertentu dengan fungsi sum().

\textgreater sum((1:10)\textless6) // banyak elemen yang kurang dari 6

\begin{verbatim}
5
\end{verbatim}

Euler memiliki operator perbandingan, seperti ``=='', yang memeriksa kesetaraan. Kita mendapatkan vektor 0 dan 1, di mana 1 berarti benar.

\textgreater t=(1:10)\^{}2; t==25 //menguji elemen2 t yang sama dengan 25 (hanya ada 1)

\begin{verbatim}
[0,  0,  0,  0,  1,  0,  0,  0,  0,  0]
\end{verbatim}

Dari vektor seperti itu, ``nonzeros'' memilih elemen bukan nol. Dalam hal ini, kita mendapatkan indeks semua elemen yang lebih besar dari 50.

\textgreater nonzeros(t\textgreater50) //indeks elemen2 t yang lebih besar daripada 50

\begin{verbatim}
[8,  9,  10]
\end{verbatim}

Tentu saja, kita dapat menggunakan vektor indeks ini untuk mendapatkan nilai yang sesuai dalam t.

\textgreater t{[}nonzeros(t\textgreater50){]} //elemen2 t yang lebih besar daripada 50

\begin{verbatim}
[64,  81,  100]
\end{verbatim}

Sebagai contoh, mari kita cari semua kuadrat dari angka 1 sampai 1000, yaitu 5 modulo 11 dan 3 modulo 13.

\textgreater t=1:1000; nonzeros(mod(t\^{}2,11)==5 \&\& mod(t\^{}2,13)==3)

\begin{verbatim}
[4,  48,  95,  139,  147,  191,  238,  282,  290,  334,  381,  425,
433,  477,  524,  568,  576,  620,  667,  711,  719,  763,  810,  854,
862,  906,  953,  997]
\end{verbatim}

EMT tidak sepenuhnya efektif untuk komputasi bilangan bulat. EMT menggunakan floating point presisi ganda secara internal. Akan tetapi, hal ini sering kali sangat berguna.

Kita dapat memeriksa bilangan prima. Mari kita cari tahu, berapa banyak kuadrat ditambah 1 yang merupakan bilangan prima.

\textgreater t=1:1000; length(nonzeros(isprime(t\^{}2+1)))

\begin{verbatim}
112
\end{verbatim}

Fungsi nonzeros() hanya bekerja untuk vektor. Untuk matriks, ada mnonzeros().

\textgreater seed(2); A=random(3,4)

\begin{verbatim}
     0.765761      0.401188      0.406347      0.267829 
      0.13673      0.390567      0.495975      0.952814 
     0.548138      0.006085      0.444255      0.539246 
\end{verbatim}

Ini mengembalikan indeks elemen, yang bukan nol.

\textgreater k=mnonzeros(A\textless0.4) //indeks elemen2 A yang kurang dari 0,4

\begin{verbatim}
            1             4 
            2             1 
            2             2 
            3             2 
\end{verbatim}

Indeks ini dapat digunakan untuk menetapkan elemen ke suatu nilai.

\textgreater mset(A,k,0) //mengganti elemen2 suatu matriks pada indeks tertentu

\begin{verbatim}
     0.765761      0.401188      0.406347             0 
            0             0      0.495975      0.952814 
     0.548138             0      0.444255      0.539246 
\end{verbatim}

Fungsi mset() juga dapat mengatur elemen-elemen pada indeks ke entri-entri matriks lain.

\textgreater mset(A,k,-random(size(A)))

\begin{verbatim}
     0.765761      0.401188      0.406347     -0.126917 
    -0.122404     -0.691673      0.495975      0.952814 
     0.548138     -0.483902      0.444255      0.539246 
\end{verbatim}

Dan dimungkinkan untuk mendapatkan elemen-elemen dalam vektor.

\textgreater mget(A,k)

\begin{verbatim}
[0.267829,  0.13673,  0.390567,  0.006085]
\end{verbatim}

Fungsi lain yang berguna adalah extrema, yang mengembalikan nilai minimal dan maksimal di setiap baris matriks dan posisinya.

\textgreater ex=extrema(A)

\begin{verbatim}
     0.267829             4      0.765761             1 
      0.13673             1      0.952814             4 
     0.006085             2      0.548138             1 
\end{verbatim}

Kita bisa menggunakan ini untuk mengekstrak nilai maksimal dalam setiap baris.

\textgreater ex{[},3{]}'

\begin{verbatim}
[0.765761,  0.952814,  0.548138]
\end{verbatim}

Ini, tentu saja, sama dengan fungsi max().

\textgreater max(A)'

\begin{verbatim}
[0.765761,  0.952814,  0.548138]
\end{verbatim}

Tetapi dengan mget(), kita dapat mengekstrak indeks dan menggunakan informasi ini untuk mengekstrak elemen-elemen pada posisi yang sama dari matriks yang lain.

\textgreater j=(1:rows(A))'\textbar ex{[},4{]}, mget(-A,j)

\begin{verbatim}
            1             1 
            2             4 
            3             1 
[-0.765761,  -0.952814,  -0.548138]
\end{verbatim}

\chapter{Fungsi Matriks Lainnya}\label{fungsi-matriks-lainnya}

Untuk membangun matriks, kita dapat menumpuk satu matriks di atas yang lain. Jika keduanya tidak memiliki jumlah kolom yang sama, kolom yang lebih pendek akan diisi dengan 0.

\textgreater v=1:3; v\_v

\begin{verbatim}
            1             2             3 
            1             2             3 
\end{verbatim}

Demikian juga, kita dapat melampirkan matriks ke matriks lain secara berdampingan, jika keduanya memiliki jumlah baris yang sama.

\textgreater A=random(3,4); A\textbar v'

\begin{verbatim}
     0.032444     0.0534171      0.595713      0.564454             1 
      0.83916      0.175552      0.396988       0.83514             2 
    0.0257573      0.658585      0.629832      0.770895             3 
\end{verbatim}

Jika keduanya tidak memiliki jumlah baris yang sama, matriks yang lebih pendek diisi dengan 0. Ada pengecualian untuk aturan ini. Bilangan real yang dilampirkan pada sebuah matriks akan digunakan sebagai kolom yang diisi dengan bilangan real tersebut.

\textgreater A\textbar1

\begin{verbatim}
     0.032444     0.0534171      0.595713      0.564454             1 
      0.83916      0.175552      0.396988       0.83514             1 
    0.0257573      0.658585      0.629832      0.770895             1 
\end{verbatim}

Dimungkinkan untuk membuat matriks vektor baris dan kolom.

\textgreater{[}v;v{]}

\begin{verbatim}
            1             2             3 
            1             2             3 
\end{verbatim}

\textgreater{[}v',v'{]}

\begin{verbatim}
            1             1 
            2             2 
            3             3 
\end{verbatim}

Tujuan utama dari hal ini adalah untuk menginterpretasikan vektor ekspresi untuk vektor kolom.

\textgreater{}``{[}x,x\^{}2{]}''(v')

\begin{verbatim}
            1             1 
            2             4 
            3             9 
\end{verbatim}

Untuk mendapatkan ukuran A, kita dapat menggunakan fungsi berikut ini.

\textgreater C=zeros(2,4); rows(C), cols(C), size(C), length(C)

\begin{verbatim}
2
4
[2,  4]
4
\end{verbatim}

yaitu menggunakan perintah length().

\textgreater length(2:10)

\begin{verbatim}
9
\end{verbatim}

Ada banyak fungsi lain yang menghasilkan matriks.

\textgreater ones(2,2)

\begin{verbatim}
            1             1 
            1             1 
\end{verbatim}

Ini juga dapat digunakan dengan satu parameter. Untuk mendapatkan vektor dengan angka selain 1, gunakan yang berikut ini.

\textgreater ones(5)*6

\begin{verbatim}
[6,  6,  6,  6,  6]
\end{verbatim}

Matriks angka acak juga dapat dibuat dengan acak (distribusi seragam) atau normal (Gauß distribution).

\textgreater random(2,2)

\begin{verbatim}
      0.66566      0.831835 
        0.977      0.544258 
\end{verbatim}

Berikut ini adalah fungsi lain yang berguna, yang merestrukturisasi elemen-elemen matriks menjadi matriks lain.

\textgreater redim(1:9,3,3) // menyusun elemen2 1, 2, 3, \ldots, 9 ke bentuk matriks 3x3

\begin{verbatim}
            1             2             3 
            4             5             6 
            7             8             9 
\end{verbatim}

Dengan fungsi berikut, kita dapat menggunakan fungsi ini dan fungsi dup untuk menulis fungsi rep(), yang mengulang sebuah vektor sebanyak n kali.

\textgreater function rep(v,n) := redim(dup(v,n),1,n*cols(v))

Mari kita coba

\textgreater rep(1:3,5)

\begin{verbatim}
[1,  2,  3,  1,  2,  3,  1,  2,  3,  1,  2,  3,  1,  2,  3]
\end{verbatim}

Fungsi multdup() menduplikasi elemen-elemen sebuah vektor.

\textgreater multdup(1:3,5), multdup(1:3,{[}2,3,2{]})

\begin{verbatim}
[1,  1,  1,  1,  1,  2,  2,  2,  2,  2,  3,  3,  3,  3,  3]
[1,  1,  2,  2,  2,  3,  3]
\end{verbatim}

Fungsi flipx() dan flipy() membalik urutan baris atau kolom dari sebuah matriks. Misalnya, fungsi flipx() membalik secara horizontal.

\textgreater flipx(1:5) //membalik elemen2 vektor baris

\begin{verbatim}
[5,  4,  3,  2,  1]
\end{verbatim}

Untuk rotasi, Euler memiliki rotleft() dan rotright().

\textgreater rotleft(1:5) // memutar elemen2 vektor baris

\begin{verbatim}
[2,  3,  4,  5,  1]
\end{verbatim}

Fungsi khusus adalah drop(v,i), yang menghapus elemen dengan indeks di i dari vektor v.

\textgreater drop(10:20,3)

\begin{verbatim}
[10,  11,  13,  14,  15,  16,  17,  18,  19,  20]
\end{verbatim}

Perhatikan bahwa vektor i dalam drop(v,i) merujuk pada indeks elemen dalam v, bukan nilai elemen. Jika Anda ingin menghapus elemen, Anda harus menemukan elemen-elemen tersebut terlebih dahulu. Fungsi indexof(v,x) dapat digunakan untuk menemukan elemen x dalam vektor terurut v.

\textgreater v=primes(50), i=indexof(v,10:20), drop(v,i)

\begin{verbatim}
[2,  3,  5,  7,  11,  13,  17,  19,  23,  29,  31,  37,  41,  43,  47]
[0,  5,  0,  6,  0,  0,  0,  7,  0,  8,  0]
[2,  3,  5,  7,  23,  29,  31,  37,  41,  43,  47]
\end{verbatim}

Seperti yang Anda lihat, tidak ada salahnya menyertakan indeks di luar jangkauan (seperti 0), indeks ganda, atau indeks yang tidak diurutkan.

\textgreater drop(1:10,shuffle({[}0,0,5,5,7,12,12{]}))

\begin{verbatim}
[1,  2,  3,  4,  6,  8,  9,  10]
\end{verbatim}

Ada beberapa fungsi khusus untuk mengatur diagonal atau menghasilkan matriks diagonal. Kita mulai dengan matriks identitas.

\textgreater A=id(5) // matriks identitas 5x5

\begin{verbatim}
            1             0             0             0             0 
            0             1             0             0             0 
            0             0             1             0             0 
            0             0             0             1             0 
            0             0             0             0             1 
\end{verbatim}

Kemudian, kami menetapkan diagonal bawah (-1) ke 1:4.

\textgreater setdiag(A,-1,1:4) //mengganti diagonal di bawah diagonal utama

\begin{verbatim}
            1             0             0             0             0 
            1             1             0             0             0 
            0             2             1             0             0 
            0             0             3             1             0 
            0             0             0             4             1 
\end{verbatim}

Perhatikan bahwa kita tidak mengubah matriks A. Kita mendapatkan sebuah matriks baru sebagai hasil dari setdiag().

Berikut adalah sebuah fungsi yang mengembalikan sebuah matriks tri-diagonal.

\textgreater function tridiag (n,a,b,c) := setdiag(setdiag(b*id(n),1,c),-1,a); \ldots{}\\
\textgreater{} tridiag(5,1,2,3)

\begin{verbatim}
            2             3             0             0             0 
            1             2             3             0             0 
            0             1             2             3             0 
            0             0             1             2             3 
            0             0             0             1             2 
\end{verbatim}

Diagonal sebuah matriks juga dapat diekstrak dari matriks. Untuk mendemonstrasikan hal ini, kami merestrukturisasi vektor 1:9 menjadi matriks 3x3.

\textgreater A=redim(1:9,3,3)

\begin{verbatim}
            1             2             3 
            4             5             6 
            7             8             9 
\end{verbatim}

Sekarang kita bisa mengekstrak diagonal.

\textgreater d=getdiag(A,0)

\begin{verbatim}
[1,  5,  9]
\end{verbatim}

Contoh: Kita dapat membagi matriks dengan diagonalnya. Bahasa matriks memperhatikan bahwa vektor kolom d diterapkan ke matriks baris demi baris.

\textgreater fraction A/d'

\begin{verbatim}
        1         2         3 
      4/5         1       6/5 
      7/9       8/9         1 
\end{verbatim}

\chapter{Vectorization}\label{vectorization}

Hampir semua fungsi di Euler juga berfungsi untuk input matriks dan vektor, kapan pun ini masuk akal. Misalnya, fungsi sqrt() menghitung akar kuadrat dari semua elemen vektor atau matriks.

\textgreater sqrt(1:3)

\begin{verbatim}
[1,  1.41421,  1.73205]
\end{verbatim}

Jadi, kamu dapat dengan mudah membuat tabel nilai. Ini adalah salah satu cara untuk memplot suatu fungsi(alternatifnya menggunakan ekspresi).

\textgreater x=1:0.01:5; y=log(x)/x\^{}2; // terlalu panjang untuk ditampikan

Dengan ini dan operator titik dua a:delta:b, vektor nilai fungsi dapat dihasilkan dengan mudah. Pada contoh berikut, kita menghasilkan vektor nilai t{[}i{]} dengan jarak 0,1 dari -1 hingga 1. Kemudian kita menghasilkan vektor nilai dari fungsi. \[s = t^3-t\]\textgreater t=-1:0.1:1; s=t\^{}3-t

\begin{verbatim}
[0,  0.171,  0.288,  0.357,  0.384,  0.375,  0.336,  0.273,  0.192,
0.099,  0,  -0.099,  -0.192,  -0.273,  -0.336,  -0.375,  -0.384,
-0.357,  -0.288,  -0.171,  0]
\end{verbatim}

EMT memperluas operator untuk skalar, vektor, dan matriks dengan cara yang jelas.

Misalnya, vektor kolom dikalikan vektor baris diperluas menjadi matriks, jika operator diterapkan. Berikut ini, v' adalah vektor yang ditransposisikan (vektor kolom).

\textgreater shortest (1:5)*(1:5)'

\begin{verbatim}
     1      2      3      4      5 
     2      4      6      8     10 
     3      6      9     12     15 
     4      8     12     16     20 
     5     10     15     20     25 
\end{verbatim}

Perhatikan, bahwa ini sangat berbeda dari produk matriks. Produk matriks dilambangkan dengan titik ``.'' di EMT.

\textgreater(1:5).(1:5)'

\begin{verbatim}
55
\end{verbatim}

Secara default, vektor baris dicetak dalam format yang ringkas.

\textgreater{[}1,2,3,4{]}

\begin{verbatim}
[1,  2,  3,  4]
\end{verbatim}

Untuk matriks operator khusus . menunjukkan perkalian matriks, dan A' menunjukkan transpos. Matriks 1x1 dapat digunakan seperti bilangan real.

\textgreater v:={[}1,2{]}; v.v', \%\^{}2

\begin{verbatim}
5
25
\end{verbatim}

Untuk mentranspos matriks kita menggunakan apostrof,

\textgreater v=1:4; v'

\begin{verbatim}
            1 
            2 
            3 
            4 
\end{verbatim}

Jadi kita dapat menghitung matriks A dikali vektor b.

\textgreater A={[}1,2,3,4;5,6,7,8{]}; A.v'

\begin{verbatim}
           30 
           70 
\end{verbatim}

Perhatikan bahwa v masih merupakan vektor baris, Jadi v'.v berbeda dengan v.v'.

\textgreater v'.v

\begin{verbatim}
            1             2             3             4 
            2             4             6             8 
            3             6             9            12 
            4             8            12            16 
\end{verbatim}

v.v' menghitung norma v kuadrat untuk vektor baris v. Hasilnya adalah vektor 1x1, yang bekerja seperti bilangan real.

\textgreater v.v'

\begin{verbatim}
30
\end{verbatim}

Ada juga fungsi norma (bersama dengan banyak fungsi lain dari Aljabar Linier).

\textgreater norm(v)\^{}2

\begin{verbatim}
30
\end{verbatim}

Operator dan fungsi mematuhi bahasa matriks Euler. Berikut ringkasan aturannya.

\begin{enumerate}
\def\labelenumi{\arabic{enumi}.}
\tightlist
\item
  Fungsi yang diterapkan ke vektor atau matriks diterapkan ke setiap elemen.
\item
  Operator yang beroperasi pada dua matriks dengan ukuran yang sama diterapkan berpasangan ke elemen matriks.
\item
  jika kedua matriks memiliki dimensi yang berbeda, keduanya diperluas dengan cara yang masuk akal, sehingga memiliki ukuran yang sama. Misalnya, nilai skalar kali vektor mengalikan nilai dengan setiap elemen vektor. Atau matriks kali vektor (dengan *, bukan.) memperluas vektor ke ukuran matriks dengan menduplikasikan.
\end{enumerate}

Berikut ini adalah kasus sederhana dengan operator\^{}.

\textgreater{[}1,2,3{]}\^{}2

\begin{verbatim}
[1,  4,  9]
\end{verbatim}

Berikut adalah kasus yang lebih rumit. Vektor baris dikalikan dengan vektor kolom mengembang keduanya dengan menduplikasi.

\textgreater v:={[}1,2,3{]}; v*v'

\begin{verbatim}
            1             2             3 
            2             4             6 
            3             6             9 
\end{verbatim}

Perhatikan bahwa produk skalar menggunakan produk matriks, bukan *!

\textgreater v.v'

\begin{verbatim}
14
\end{verbatim}

Ada banyak fungsi matriks. Kami memberikan daftar singkat. Anda harus berkonsultasi dengan dokumentasi untuk informasi lebih lanjut tentang perintah ini.

sum,prod menghitung jumlah dan produk dari baris\\
cumsum,cumprod melakukan hal yang sama secara kumulatif menghitung nilai ekstrem dari setiap baris\\
extrema mengembalikan vektor dengan informasi ekstrim\\
diag(A,i) mengembalikan diagonal ke-i\\
setdiag(A,i,v) mengatur diagonal ke-i\\
id(n) matriks identitas\\
det(A) penentu\\
charpoly(A) polinomial karakteristik\\
nilai eigen(A) nilai eigen.

\textgreater v*v, sum(v*v), cumsum(v*v)

\begin{verbatim}
[1,  4,  9]
14
[1,  5,  14]
\end{verbatim}

Operator : menghasilkan vektor baris spasi yang sama, opsional dengan ukuran langkah.

\textgreater1:4, 1:2:10

\begin{verbatim}
[1,  2,  3,  4]
[1,  3,  5,  7,  9]
\end{verbatim}

Untuk menggabungkan matriks dan vektor, terdapat operator ``\textbar{}'' dan ``\_''.

\textgreater{[}1,2,3{]}\textbar{[}4,5{]}, {[}1,2,3{]}\_1

\begin{verbatim}
[1,  2,  3,  4,  5]
            1             2             3 
            1             1             1 
\end{verbatim}

Elemen-elemen dari sebuah matriks disebut dengan ``A{[}i,j{]}''.

\textgreater A:={[}1,2,3;4,5,6;7,8,9{]}; A{[}2,3{]}

\begin{verbatim}
6
\end{verbatim}

Untuk vektor baris atau kolom, v{[}i{]} adalah elemen ke-i dari vektor. Untuk matriks, ini mengembalikan baris ke-i lengkap dari matriks.

\textgreater v:={[}2,4,6,8{]}; v{[}3{]}, A{[}3{]}

\begin{verbatim}
6
[7,  8,  9]
\end{verbatim}

Indeks juga bisa menjadi vektor baris dari indeks. : menunjukkan semua indeks.

\textgreater v{[}1:2{]}, A{[}:,2{]}

\begin{verbatim}
[2,  4]
            2 
            5 
            8 
\end{verbatim}

Bentuk singkat untuk : adalah menghilangkan indeks sepenuhnya.

\textgreater A{[},2:3{]}

\begin{verbatim}
            2             3 
            5             6 
            8             9 
\end{verbatim}

Untuk tujuan vektorisasi, elemen matriks dapat diakses seolah-olah mereka adalah vektor.

\textgreater A\{4\}

\begin{verbatim}
4
\end{verbatim}

Matriks juga dapat diratakan, menggunakan fungsi redim(). Ini diimplementasikan dalam fungsi flatten().

\textgreater redim(A,1,prod(size(A))), flatten(A)

\begin{verbatim}
[1,  2,  3,  4,  5,  6,  7,  8,  9]
[1,  2,  3,  4,  5,  6,  7,  8,  9]
\end{verbatim}

Untuk menggunakan matriks untuk tabel, mari kita reset ke format default, dan menghitung tabel nilai sinus dan kosinus. Perhatikan bahwa sudut dalam radian secara default.

\textgreater defformat; w=0°:45°:360°; w=w'; deg(w)

\begin{verbatim}
            0 
           45 
           90 
          135 
          180 
          225 
          270 
          315 
          360 
\end{verbatim}

Sekarang kita menambahkan kolom ke matriks.

\textgreater M = deg(w)\textbar w\textbar cos(w)\textbar sin(w)

\begin{verbatim}
            0             0             1             0 
           45      0.785398      0.707107      0.707107 
           90        1.5708             0             1 
          135       2.35619     -0.707107      0.707107 
          180       3.14159            -1             0 
          225       3.92699     -0.707107     -0.707107 
          270       4.71239             0            -1 
          315       5.49779      0.707107     -0.707107 
          360       6.28319             1             0 
\end{verbatim}

Dengan menggunakan bahasa matriks, kita dapat menghasilkan beberapa tabel dari beberapa fungsi sekaligus. Dalam contoh berikut, kita menghitung t{[}j{]}\^{}i untuk i dari 1 hingga n.~Kami mendapatkan matriks, di mana setiap baris adalah tabel t\^{}i untuk satu i. Yaitu, matriks memiliki elemen \[a_{i,j} = t_j^i, \quad 1 \le j \le 101, \quad 1 \le i \le n\]Fungsi yang tidak berfungsi untuk input vektor harus ``vectorized''. Ini dapat dicapai dengan kata kunci ``map'' dalam definisi fungsi.Kemudian fungsi tersebut akan dievaluasi untuk setiap elemen dari parameter vektor. Integrasi numerik terintegrasi() hanya berfungsi untuk batas interval skalar. Jadi kita perlu membuat vektor.

\textgreater function map f(x) := integrate(``x\^{}x'',1,x)

Kata kunci ``map'' membuat vektor fungsi. Fungsinya sekarang akan bekerja untuk vektor bilangan.

\textgreater f({[}1:5{]})

\begin{verbatim}
[0,  2.05045,  13.7251,  113.336,  1241.03]
\end{verbatim}

\chapter{Sub-Matrices dan Matrix-Elements}\label{sub-matrices-dan-matrix-elements}

Untuk mengakses elemen matriks, gunakan notasi braket.

\textgreater A={[}1,2,3;4,5,6;7,8,9{]}, A{[}2,2{]}

\begin{verbatim}
            1             2             3 
            4             5             6 
            7             8             9 
5
\end{verbatim}

Kita dapat mengakses satu baris matriks yang lengkap.

\textgreater A{[}2{]}

\begin{verbatim}
[4,  5,  6]
\end{verbatim}

Dalam kasus vektor baris atau kolom, ini mengembalikan elemen vektor.

\textgreater v=1:3; v{[}2{]}

\begin{verbatim}
2
\end{verbatim}

Untuk memastikan, Anda mendapatkan baris pertama untuk matriks 1xn dan mxn, tentukan semua kolom menggunakan indeks kedua kosong.

\textgreater A{[}2,{]}

\begin{verbatim}
[4,  5,  6]
\end{verbatim}

Jika indeks adalah vektor indeks, Euler akan mengembalikan baris matriks yang sesuai. Di sini kita ingin baris pertama dan kedua dari A.

\begin{verbatim}
            1             2             3 
            4             5             6 
\end{verbatim}

Kita bahkan dapat menyusun ulang A menggunakan vektor indeks. Tepatnya, kami tidak mengubah A disini, tetapi menghitung versi A yang disusun ulang.

\begin{verbatim}
            7             8             9 
            4             5             6 
            1             2             3 
\end{verbatim}

Trik indeks bekerja dengan kolom juga. Contoh ini memilih semua baris A dan kolom kedua dan ketiga.

\textgreater A{[}1:3,2:3{]}

\begin{verbatim}
            2             3 
            5             6 
            8             9 
\end{verbatim}

Untuk singkatan ``:'' menunjukkan semua indeks baris atau kolom.

\textgreater A{[}:,3{]}

\begin{verbatim}
            3 
            6 
            9 
\end{verbatim}

Atau, biarkan indeks pertama kosong.

\textgreater A{[},2:3{]}

\begin{verbatim}
            2             3 
            5             6 
            8             9 
\end{verbatim}

Kita juga bisa mendapatkan baris terakhir dari A.

\textgreater A{[}-1{]}

\begin{verbatim}
[7,  8,  9]
\end{verbatim}

Sekarang mari kita ubah elemen A dengan menetapkan submatriks A ke beberapa nilai. Ini sebenarnya mengubah matriks A yang disimpan.

\textgreater A{[}1,1{]}=4

\begin{verbatim}
            4             2             3 
            4             5             6 
            7             8             9 
\end{verbatim}

Kita juga dapat menetapkan nilai pada deretan A.

\textgreater A{[}1{]}={[}-1,-1,-1{]}

\begin{verbatim}
           -1            -1            -1 
            4             5             6 
            7             8             9 
\end{verbatim}

Kami bahkan dapat menetapkan sub-matriks jika memiliki ukuran yang tepat.

\textgreater A{[}1:2,1:2{]}={[}5,6;7,8{]}

\begin{verbatim}
            5             6            -1 
            7             8             6 
            7             8             9 
\end{verbatim}

Selain itu, beberapa jalan pintas diperbolehkan.

\textgreater A{[}1:2,1:2{]}=0

\begin{verbatim}
            0             0            -1 
            0             0             6 
            7             8             9 
\end{verbatim}

Peringatan: Indeks di luar batas mengembalikan matriks kosong, atau pesan kesalahan, tergantung pada pengaturan sistem. Standarnya adalah pesan kesalahan. Ingat, bagaimanapun, bahwa indeks negatif dapat digunakan untuk mengakses elemen matriks yang dihitung dari akhir.

\textgreater A{[}4{]}

\begin{verbatim}
Row index 4 out of bounds!
Error in:
A[4] ...
    ^
\end{verbatim}

\chapter{Menyortir dan mengacak}\label{menyortir-dan-mengacak}

Fungsi sort() mengurutkan vektor baris

\textgreater sort({[}5,6,4,8,1,9{]})

\begin{verbatim}
[1,  4,  5,  6,  8,  9]
\end{verbatim}

Seringkali perlu untuk mengetahui indeks dari vektor yang diurutkan dalam vektor aslinya. Ini dapat digunakan untuk menyusun ulang vektor lain dengan cara yang sama. Mari kita mengacak vektor.

\textgreater v=shuffle(1:10)

\begin{verbatim}
[4,  5,  10,  6,  8,  9,  1,  7,  2,  3]
\end{verbatim}

Indeks berisi urutan yang tepat dari v.

\textgreater\{vs,ind\}=sort(v); v{[}ind{]}

\begin{verbatim}
[1,  2,  3,  4,  5,  6,  7,  8,  9,  10]
\end{verbatim}

Hal ini juga berlaku untuk vektor string.

\textgreater s={[}``a'',``d'',``e'',``a'',``aa'',``e''{]}

\begin{verbatim}
a
d
e
a
aa
e
\end{verbatim}

\textgreater\{ss,ind\}=sort(s); ss

\begin{verbatim}
a
a
aa
d
e
e
\end{verbatim}

Seperti yang Anda lihat, posisi entri ganda agak acak.

\textgreater ind

\begin{verbatim}
[4,  1,  5,  2,  6,  3]
\end{verbatim}

Fungsi unique mengembalikan daftar terurut dari elemen unik sebuah vektor.

\textgreater intrandom(1,10,10), unique(\%)

\begin{verbatim}
[4,  4,  9,  2,  6,  5,  10,  6,  5,  1]
[1,  2,  4,  5,  6,  9,  10]
\end{verbatim}

Hal ini juga berlaku untuk vektor string.

\textgreater unique(s)

\begin{verbatim}
a
aa
d
e
\end{verbatim}

\chapter{Aljabar Linier}\label{aljabar-linier}

EMT memiliki banyak fungsi untuk menyelesaikan sistem linier, sistem sparse, atau masalah regresi.

Untuk sistem linier Ax=b,dengan A adalah matriks koefisien, x adalah vektor solusi yang ingin kita cari dan b adalah vektor hasil yang diberikan.Anda dapat menggunakan algoritma Gauss, matriks invers atau kecocokan linier.

Operator A\b menggunakan versi algoritma Gauss.

Operator backslash ~digunakan untuk menyelesaikan sistem persamaan linier ini. Ketika menulis A\b, perangkat lunak akan menghitung solusi yang memenuhi persamaan Ax=b. Operator ini secara otomatis menggunakan algoritma eliminasi Gauss atau metode numerik serupa untuk menemukan solusi.

\textgreater A={[}1,2;3,4{]}; b={[}5;6{]}; A\textbackslash b

\begin{verbatim}
           -4 
          4.5 
\end{verbatim}

Untuk contoh lain, kami membuat matriks 200x200 dan jumlah barisnya. Kemudian kita selesaikan Ax=b menggunakan matriks invers. Kami mengukur kesalahan sebagai deviasi maksimal semua elemen dari 1,yang tentu saja merupakan solusi yang benar.

\textgreater A=normal(200,200); b=sum(A); longest totalmax(abs(inv(A).b-1))

\begin{verbatim}
  8.790745908981989e-13 
\end{verbatim}

Jika sistem tidak memiliki solusi, kecocokan linier meminimalkan norma kesalahan Ax-b.

\textgreater A={[}1,2,3;4,5,6;7,8,9{]}

\begin{verbatim}
            1             2             3 
            4             5             6 
            7             8             9 
\end{verbatim}

Determinan matriks ini adalah 0.

\textgreater det(A)

\begin{verbatim}
0
\end{verbatim}

\chapter{Matriks Simbolik}\label{matriks-simbolik}

Maxima memiliki matriks simbolis. Tentu saja, Maxima dapat digunakan untuk masalah aljabar linier sederhana seperti itu. Kita dapat mendefinisikan matriks untuk Euler dan Maxima dengan \&:=, dan kemudian menggunakannya dalam ekspresi simbolis. Bentuk {[}\ldots{]} biasa untuk mendefinisikan matriks dapat digunakan di Euler untuk mendefinisikan matriks simbolik.

\textgreater A \&= {[}a,1,1;1,a,1;1,1,a{]}; \(A\)\(\begin{pmatrix}a & 1 & 1 \\ 1 & a & 1 \\ 1 & 1 & a \\ \end{pmatrix}\)\(\>\)\&det(A), \(&factor(%)
\)\(\left(a-1\right)^2\,\left(a+2\right)\)\$\pandocbounded{\includegraphics[keepaspectratio]{images/EMT untuk Perhitungan Aljabar_Isni Azizah Utami_23030630016-089.png}}

\textgreater{}\(&invert(A) with a=0\)\(\begin{pmatrix}-\frac{1}{2} & \frac{1}{2} & \frac{1}{2} \\ \frac{1  }{2} & -\frac{1}{2} & \frac{1}{2} \\ \frac{1}{2} & \frac{1}{2} & -  \frac{1}{2} \\ \end{pmatrix}\)\$\textgreater A \&= {[}1,a;b,2{]}; \(A\)\(\begin{pmatrix}1 & a \\ b & 2 \\ \end{pmatrix}\)\$dengan

\begin{enumerate}
\def\labelenumi{\arabic{enumi}.}
\tightlist
\item
  \texttt{det(A)} menghitung determinan matriks A.
\item
  \texttt{factor(\%\ )} memberikan faktor-faktor dari determinan.
\item
  \texttt{invert(A)} memberikan invers dari matriks A.
\end{enumerate}

Seperti semua variabel simbolik, matriks ini dapat digunakan dalam ekspresi simbolik lainnya.

\textgreater\$\&det(A-x*ident(2)), \$\&solve(\%,x)

\$\left[ x=\frac{3-\sqrt{4\,a\,b+1}}{2} , x=\frac{\sqrt{4\,a\,b+1}+3  }{2} \right] \$\$ \pandocbounded{\includegraphics[keepaspectratio]{images/EMT untuk Perhitungan Aljabar_Isni Azizah Utami_23030630016-093.png}}

Nilai eigen juga dapat dihitung secara otomatis. Hasilnya adalah vektor dengan dua vektor nilai eigen dan multiplisitas.

\textgreater\$\&eigenvalues({[}a,1;1,a{]})

\$\left[ \left[ a-1 , a+1 \right]  , \left[ 1 , 1 \right]  \right{]} \$\$

Untuk mengekstrak vektor eigen tertentu perlu pengindeksan yang cermat.

\textgreater\$\&eigenvectors({[}a,1;1,a{]}), \&\%{[}2{]}{[}1{]}{[}1{]}

\$\left[ \left[ \left[ a-1 , a+1 \right]  , \left[ 1 , 1 \right]    \right{]} , \left[ \left[ \left[ 1 , -1 \right]  \right{]} , \left[   \left[ 1 , 1 \right]  \right{]} \right{]} \right{]} \$\$ {[}1, - 1{]}

Matriks simbolik dapat dievaluasi dalam Euler secara numerik sepert ekspresi simbolik lainnya.

\textgreater A(a=4,b=5)

\begin{verbatim}
            1             4 
            5             2 
\end{verbatim}

Dalam ekspresi simbolik, gunakan ``with''.

\textgreater{}\(&A with [a=4,b=5]\)\(\begin{pmatrix}1 & 4 \\ 5 & 2 \\ \end{pmatrix}\)\$Akses ke baris matriks simbolik bekerja seperti halnya dengan matriks numerik.

\textgreater\$\&A{[}1{]}

\$\left[ 1 , a \right] \$\$

Ekspresi simbolis dapat berisi tugas. Dan itu mengubah matriks A.

\textgreater\&A{[}1,1{]}:=t+1; \(&A\)\(\begin{pmatrix}t+1 & a \\ b & 2 \\ \end{pmatrix}\)\$Terdapat fungsi-fungsi simbolik dalam Maxima untuk membuat vektor dan matriks. Untuk hal ini, lihat dokumentasi Maxima atau tutorial tentang Maxima di EMT.

\textgreater v \&= makelist(1/(i+j),i,1,3); \$v

\$\left[ \frac{1}{j+1} , \frac{1}{j+2} , \frac{1}{j+3} \right] \$\$

\textgreater B \&:= {[}1,2;3,4{]}; \$B, \(&invert(B)\)\(\begin{pmatrix}-2 & 1 \\ \frac{3}{2} & -\frac{1}{2} \\   \end{pmatrix}\)\$\pandocbounded{\includegraphics[keepaspectratio]{images/EMT untuk Perhitungan Aljabar_Isni Azizah Utami_23030630016-101.png}}

Hasilnya dapat dievaluasi secara numerik dalam Euler. Untuk informasi lebih lanjut tentang Maxima, lihat pengantar Maxima.

\textgreater\$\&invert(B)()

\begin{verbatim}
           -2             1 
          1.5          -0.5 
\end{verbatim}

Euler juga memiliki fungsi xinv() yang kuat, yang membuat upaya lebih besar dan mendapatkan hasil yang lebih tepat. Perhatikan, bahwa dengan \&:= matriks B telah didefinisikan sebagai simbolik dalam ekspresi simbolik dan sebagai numerik dalam ekspresi numerik. Jadi kita bisa menggunakannya di sini.

\textgreater longest B.xinv(B)

\begin{verbatim}
                      1                       0 
                      0                       1 
\end{verbatim}

Misalnya. nilai eigen dari A dapat dihitung secara numerik.

\textgreater A={[}1,2,3;4,5,6;7,8,9{]}; real(eigenvalues(A))

\begin{verbatim}
[16.1168,  -1.11684,  0]
\end{verbatim}

Atau secara simbolis. Lihat tutorial tentang Maxima untuk detailnya.

\textgreater\$\&eigenvalues((\textbf{A?}))

\$\left[ \left[ \frac{15-3\,\sqrt{33}}{2} , \frac{3\,\sqrt{33}+15}{2}   , 0 \right]  , \left[ 1 , 1 , 1 \right]  \right{]} \$\$

\chapter{Nilai Numerik dalam Ekspresi simbolis}\label{nilai-numerik-dalam-ekspresi-simbolis}

Ekspresi simbolis hanyalah string yang berisi ekspresi. Jika kita ingin mendefinisikan nilai baik untuk ekspresi simbolik maupun ekspresi numerik, kita harus menggunakan ``\&:=''.

\textgreater A \&:= {[}1,pi;4,5{]}

\begin{verbatim}
            1       3.14159 
            4             5 
\end{verbatim}

Masih ada perbedaan antara bentuk numerik dan simbolik. Saat mentransfer matriks ke bentuk simbolis, pendekatan fraksional untuk real akan digunakan.

\textgreater{}\(&A\)\(\begin{pmatrix}1 & \frac{1146408}{364913} \\ 4 & 5 \\ \end{pmatrix}\)\$Untuk menghindarinya, ada fungsi ``mxmset(variable)''.

\textgreater mxmset(A); \$\&A

\[\begin{pmatrix}1 & 3.141592653589793 \\ 4 & 5 \\ \end{pmatrix}\]Maxima juga dapat menghitung dengan angka floating point, dan bahkan dengan angka floating besar dengan 32 digit. Namun, evaluasinya jauh lebih lambat.

\textgreater\$\&bfloat(sqrt(2)), \$\&float(sqrt(2))

\[1.414213562373095\]\pandocbounded{\includegraphics[keepaspectratio]{images/EMT untuk Perhitungan Aljabar_Isni Azizah Utami_23030630016-106.png}}

Ketepatan angka floating point besar dapat diubah.

\textgreater\&fpprec:=100; \&bfloat(pi)

\begin{verbatim}
        3.14159265358979323846264338327950288419716939937510582097494\
4592307816406286208998628034825342117068b0
\end{verbatim}

Variabel numerik dapat digunakan dalam ekspresi simbolis apa pun menggunakan ``(\textbf{var?})''. Perhatikan bahwa ini hanya diperlukan, jika variabel telah didefinisikan dengan ``:='' atau ``='' sebagai variabel numerik.

\textgreater B:={[}1,pi;3,4{]}; \$\&det((\textbf{B?}))

\[-5.424777960769379\]

\chapter{Demo - Suku Bunga}\label{demo---suku-bunga}

Di bawah ini, kami menggunakan Euler Math Toolbox (EMT) untuk perhitungan suku bunga. Kami melakukannya secara numerik dan simbolis untuk menunjukkan kepada Anda bagaimana Euler dapat digunakan untuk memecahkan masalah kehidupan nyata.

Asumsikan Anda memiliki modal awal 5000 (katakanlah dalam dolar).

\textgreater K=5000

\begin{verbatim}
5000
\end{verbatim}

Sekarang kita asumsikan tingkat bunga 3\% per tahun. Mari kita tambahkan satu tarif sederhana dan hitung hasilnya.

\textgreater K*1.03

\begin{verbatim}
5150
\end{verbatim}

Euler akan memahami sintaks berikut juga.

\textgreater K+K*3\%

\begin{verbatim}
5150
\end{verbatim}

Tetapi lebih mudah menggunakan faktornya.

\textgreater q=1+3\%, K*q

\begin{verbatim}
1.03
5150
\end{verbatim}

Selama 10 tahun, kita cukup mengalikan faktornya dan mendapatkan nilai akhir dengan suku bunga majemuk.

\textgreater K*q\^{}10

\begin{verbatim}
6719.58189672
\end{verbatim}

Untuk tujuan kita, kita dapat mengatur format menjadi 2 digit setelah titik desimal.

\textgreater format(12,2); K*q\^{}10

\begin{verbatim}
    6719.58 
\end{verbatim}

Mari kita cetak yang dibulatkan menjadi 2 digit dalam kalimat lengkap.

\textgreater{}``Starting from'' + K + ``\$ you get'' + round(K*q\^{}10,2) + ``\$.''

\begin{verbatim}
Starting from 5000$ you get 6719.58$.
\end{verbatim}

Bagaimana jika kita ingin mengetahui hasil antara dari tahun 1 sampai tahun 9? Untuk ini, bahasa matriks Euler sangat membantu. Anda tidak harus menulis loop, tetapi cukup masukkan.

\textgreater K*q\^{}(0:10)

\begin{verbatim}
Real 1 x 11 matrix

    5000.00     5150.00     5304.50     5463.64     ...
\end{verbatim}

Bagaimana keajaiban ini bekerja? Pertama ekspresi 0:10 mengembalikan vektor bilangan bulat.

\textgreater short 0:10

\begin{verbatim}
[0,  1,  2,  3,  4,  5,  6,  7,  8,  9,  10]
\end{verbatim}

Kemudian semua operator dan fungsi dalam Euler dapat diterapkan pada elemen vektor untuk elemen. jadi

\textgreater short q\^{}(0:10)

\begin{verbatim}
[1,  1.03,  1.0609,  1.0927,  1.1255,  1.1593,  1.1941,  1.2299,
1.2668,  1.3048,  1.3439]
\end{verbatim}

adalah vektor faktor qˆ0 sampai qˆ10. Ini dikalikan dengan K, dan kami mendapatkan vektor nilai.

\textgreater VK=K*q\^{}(0:10);

Tentu saja, cara realistis untuk menghitung suku bunga ini adalah dengan membulatkan ke sen terdekat setelah setiap tahun. Mari kita tambahkan fungsi untuk ini.

\textgreater function oneyear (K) := round(K*q,2)

Mari kita bandingkan kedua hasil tersebut, dengan dan tanpa pembulatan.

\textgreater longest oneyear(1234.57), longest 1234.57*q

\begin{verbatim}
                1271.61 
              1271.6071 
\end{verbatim}

Sekarang tidak ada rumus sederhana untuk tahun ke-n, dan kita harus mengulang selama bertahun-tahun. Euler memberikan banyak solusi untuk ini.

Cara termudah adalah iterasi fungsi, yang mengulangi fungsi tertentu beberapa kali.

\textgreater VKr=iterate(``oneyear'',5000,10)

\begin{verbatim}
Real 1 x 11 matrix

    5000.00     5150.00     5304.50     5463.64     ...
\end{verbatim}

Kami dapat mencetaknya dengan cara yang ramah, menggunakan format kami dengan tempat desimal tetap.

\textgreater VKr'

\begin{verbatim}
    5000.00 
    5150.00 
    5304.50 
    5463.64 
    5627.55 
    5796.38 
    5970.27 
    6149.38 
    6333.86 
    6523.88 
    6719.60 
\end{verbatim}

Untuk mendapatkan elemen tertentu dari vektor, kami menggunakan indeks dalam tanda kurung siku.

\textgreater VKr{[}2{]}, VKr{[}1:3{]}

\begin{verbatim}
    5150.00 
    5000.00     5150.00     5304.50 
\end{verbatim}

Anehnya, kita juga bisa menggunakan vektor indeks. Ingat bahwa 1:3 menghasilkan vektor {[}1,2,3{]}.

Mari kita bandingkan elemen terakhir dari nilai yang dibulatkan dengan nilai penuh.

\textgreater VKr{[}-1{]}, VK{[}-1{]}

\begin{verbatim}
    6719.60 
    6719.58 
\end{verbatim}

Perbedaannya sangat kecil.

\chapter{Memecahkan Persamaan}\label{memecahkan-persamaan}

Sekarang kita mengambil fungsi yang lebih maju, yang menambahkan tingkat uang tertentu setiap tahun.

\textgreater function onepay (K) := K*q+R

Kita tidak perlu menentukan q atau R untuk definisi fungsi. Hanya jika kita menjalankan perintah, kita harus mendefinisikan nilai-nilai ini. Kami memilih R=200.

\textgreater R=200; iterate(``onepay'',5000,10)

\begin{verbatim}
Real 1 x 11 matrix

    5000.00     5350.00     5710.50     6081.82     ...
\end{verbatim}

Bagaimana jika kita menghapus jumlah yang sama setiap tahun?

\textgreater R=-200; iterate(``onepay'',5000,10)

\begin{verbatim}
Real 1 x 11 matrix

    5000.00     4950.00     4898.50     4845.45     ...
\end{verbatim}

Kami melihat bahwa uang berkurang. Jelas, jika kita hanya mendapatkan 150 bunga di tahun pertama, tetapi menghapus 200, kita kehilangan uang setiap tahun.

Bagaimana kita bisa menentukan berapa tahun uang itu akan bertahan? Kita harus menulis loop untuk ini. Cara termudah adalah dengan iterasi cukup lama.

\textgreater VKR=iterate(``onepay'',5000,50)

\begin{verbatim}
Real 1 x 51 matrix

    5000.00     4950.00     4898.50     4845.45     ...
\end{verbatim}

Dengan menggunakan bahasa matriks, kita dapat menentukan nilai negatif pertama dengan cara berikut.

\textgreater min(nonzeros(VKR\textless0))

\begin{verbatim}
      48.00 
\end{verbatim}

Alasan untuk ini adalah bahwa nonzeros(VKR\textless0) mengembalikan vektor indeks i, di mana VKR{[}i{]}\textless0,dan min menghitung indeks minimal.

Karena vektor selalu dimulai dengan indeks 1, jawabannya adalah 47 tahun.

Fungsi iterate() memiliki satu trik lagi. Itu bisa mengambil kondisi akhir sebagai argumen. Kemudian akan mengembalikan nilai dan jumlah iterasi.

\textgreater\{x,n\}=iterate(``onepay'',5000,till=``x\textless0''); x, n,

\begin{verbatim}
     -19.83 
      47.00 
\end{verbatim}

Mari kita coba menjawab pertanyaan yang lebih ambigu. Asumsikan kita tahu bahwa nilainya adalah 0 setelah 50 tahun. Apa yang akan menjadi tingkat bunga?

Ini adalah pertanyaan yang hanya bisa dijawab dengan angka. Di bawah ini, kita akan mendapatkan formula yang diperlukan. Kemudian Anda akan melihat bahwa tidak ada formula yang mudah untuk tingkat bunga. Tapi untuk saat ini, kami bertujuan untuk solusi numerik.

Langkah pertama adalah mendefinisikan fungsi yang melakukan iterasi sebanyak n kali. Kami menambahkan semua parameter ke fungsi ini.

\textgreater function f(K,R,P,n) := iterate(``x*(1+P/100)+R'',K,n;P,R){[}-1{]}

Iterasinya sama seperti di atas. \[x_{n+1} = x_n \cdot \left(1+ \frac{P}{100}\right) + R\] Tapi kami tidak lagi menggunakan nilai global R dalam ekspresi kami.

Fungsi seperti iterate() memiliki trik khusus di Euler. Anda dapat meneruskan nilai variabel dalam ekspresi sebagai parameter titik koma. Dalam hal ini P dan R.

Selain itu, kami hanya tertarik pada nilai terakhir. Jadi kita ambil indeks {[}-1{]}.

Mari kita coba tes.

\textgreater f(5000,-200,3,47)

\begin{verbatim}
     -19.83 
\end{verbatim}

Sekarang kita bisa menyelesaikan masalah kita.

\textgreater solve(``f(5000,-200,x,50)'',3)

\begin{verbatim}
       3.15 
\end{verbatim}

Rutin memecahkan memecahkan ekspresi=0 untuk variabel x. Jawabannya adalah 3,15\% per tahun. Kami mengambil nilai awal 3\% untuk algoritma. Fungsi solve() selalu membutuhkan nilai awal. Kita dapat menggunakan fungsi yang sama untuk menyelesaikan pertanyaan berikut:

Berapa banyak yang dapat kita keluarkan per tahun sehingga modal awal habis setelah 20 tahun dengan asumsi tingkat bunga 3\% per tahun.

\textgreater solve(``f(5000,x,3,20)'',-200)

\begin{verbatim}
    -336.08 
\end{verbatim}

Perhatikan bahwa Anda tidak dapat memecahkan jumlah tahun, karena fungsi kami mengasumsikan n sebagai nilai integer.

\chapter{Solusi Simbolik untuk Masalah Suku Bunga}\label{solusi-simbolik-untuk-masalah-suku-bunga}

Kita dapat menggunakan bagian simbolik dari Euler untuk mempelajari masalah tersebut. Pertama kita mendefinisikan fungsi onepay() kita secara simbolis.

\textgreater function op(K) \&= K*q+R; \(&op(K)\)\(R+q\,K\)\$Kita sekarang dapat mengulangi ini.

\textgreater\$\&op(op(op(op(K)))), \(&expand(%)
\)\(q^3\,R+q^2\,R+q\,R+R+q^4\,K\)\$\pandocbounded{\includegraphics[keepaspectratio]{images/EMT untuk Perhitungan Aljabar_Isni Azizah Utami_23030630016-111.png}}

Kami melihat sebuah pola. Setelah n periode yang kita miliki\[K_n = q^n K + R (1+q+\ldots+q^{n-1}) = q^n K + \frac{q^n-1}{q-1} R\]Rumusnya adalah rumus untuk jumlah geometri, yang diketahui Maxima.

\textgreater\&sum(q\^{}k,k,0,n-1); \(& % = ev(%,simpsum)
\)\(\sum_{k=0}^{n-1}{q^{k}}=\frac{q^{n}-1}{q-1}\)\$Ini agak rumit. Jumlahnya dievaluasi dengan bendera ''simpsum'' untuk menguranginya menjadi hasil bagi. Mari kita membuat fungsi untuk ini.

Rumusnya adalah rumus untuk jumlah geometri, yang diketahui Maxima.

\textgreater function fs(K,R,P,n) \&= (1+P/100)\^{}n*K + ((1+P/100)\^{}n-1)/(P/100)*R; \(&fs(K,R,P,n)\)\(\frac{100\,\left(\left(\frac{P}{100}+1\right)^{n}-1\right)\,R}{P}+K  \,\left(\frac{P}{100}+1\right)^{n}\)\$Fungsi tersebut melakukan hal yang sama seperti fungsi f kita sebelumnya. Tapi itu lebih efektif.

\textgreater longest f(5000,-200,3,47), longest fs(5000,-200,3,47)

\begin{verbatim}
     -19.82504734650985 
     -19.82504734652684 
\end{verbatim}

Kita sekarang dapat menggunakannya untuk menanyakan waktu n.~Kapan modal kita habis? Dugaan awal kami adalah 30 tahun.

fungsi untuk ini. Rumusnya adalah rumus untuk jumlah geometri, yang diketahui Maxima.

\textgreater solve(``fs(5000,-330,3,x)'',30)

\begin{verbatim}
      20.51 
\end{verbatim}

Jawaban ini mengatakan bahwa itu akan menjadi negatif setelah 21 tahun.

Kita juga dapat menggunakan sisi simbolis Euler untuk menghitung formula pembayaran. Asumsikan kita mendapatkan pinjaman sebesar K, dan membayar n pembayaran sebesar R (dimulai setelah tahun pertama) meninggalkan sisa hutang sebesar Kn (pada saat pembayaran terakhir). Rumus untuk ini jelas.

\textgreater equ \&= fs(K,R,P,n)=Kn; \(&equ\)\(\frac{100\,\left(\left(\frac{P}{100}+1\right)^{n}-1\right)\,R}{P}+K  \,\left(\frac{P}{100}+1\right)^{n}={\it Kn}\)\(Biasanya rumus ini diberikan dalam bentuk.\)\(i = \frac{P}{100}\)\$\textgreater equ \&= (equ with P=100*i); \(&equ\)\(\frac{\left(\left(i+1\right)^{n}-1\right)\,R}{i}+\left(i+1\right)^{  n}\,K={\it Kn}\)\$Kita dapat memecahkan tingkat R secara simbolis.

\textgreater\$\&solve(equ,R)

\$\left[ R=\frac{i\,{\it Kn}-i\,\left(i+1\right)^{n}\,K}{\left(i+1  \right)^{n}-1} \right] \$\$

Seperti yang Anda lihat dari rumus, fungsi ini mengembalikan kesalahan titik mengambang untuk i=0. Euler tetap merencanakannya. Tentu saja, kami memiliki batas beriku

\textgreater{}\(&limit(R(5000,0,x,10),x,0)\)\(\lim_{x\rightarrow 0}{R\left(5000 , 0 , x , 10\right)}\)\$Jelas, tanpa bunga kita harus membayar kembali 10 tarif 500. Persamaan juga dapat diselesaikan untuk n.~Kelihatannya lebih bagus, jika kita menerapkan beberapa penyederhanaan untuk itu.

\textgreater fn \&= solve(equ,n) \textbar{} ratsimp; \$\&fn

\$\left[ n=\frac{\log \left(\frac{R+i\,{\it Kn}}{R+i\,K}\right)}{  \log \left(i+1\right)} \right] \$\$

\chapter{Latihan Soal}\label{latihan-soal}

Nama: Isni Azizah Utami

Kelas: Matematika B

NIM : 23030630016

\section{R.2 Exercise set}\label{r.2-exercise-set}

Soal No 49 Menyederhanakan:\[\left(\frac{24a^{10}b^{-8}c^7}{12a^6b^{-3}c^5}\right)^{-5}\]\textgreater{}\(& ((24\*a^(10)\*b^(-8)\*c^7)/(12\*a^(6)\*b^(-3)\*c^5))^(-5)\)\(\frac{b^{25}}{32\,a^{20}\,c^{10}}\)\(Soal No 50
Menyederhanakan:\)\(\left(\frac{125p^{12}q^{-14}r^{22}}{25p^8q^6r^{-15}}\right)^{-4}\)\(\>\)\& ((125*p\textsuperscript{(12)*q}(-14)*r\textsuperscript{(22))/(25*p}(8)*q\textsuperscript{(6)*r}(-15)))\^{}-4 \[\frac{q^{80}}{625\,p^{16}\,r^{148}}\]Soal No 90 calculate\[2^6*2^{-3}/2^{10}/2^{-8}\]\textgreater\$\& 2\textsuperscript{(6)*2}(-3)/2\textsuperscript{10/2}-8

\(2\)

Soal No 91 Calculate \[\left(\frac{4(8-6)^2-4\times3+2\times8}{3^1+9^0}\right)\]\textgreater{}\(& (4\*(8-6)^2 - 4\*3 + 2\*8)/(3^1+19^0)\)\(5\)\(Soal No 92
Calculate\)\(\left(\frac{[4(8-6)^2+4](3-2*8)}{2^2(2^5+5)}\right)\)\$\textgreater((4*(8-6)\textsuperscript{2+4)*3-2*8)/(3}1+9\^{}0)

\begin{verbatim}
11
\end{verbatim}

\section{R.3 Exercise Set}\label{r.3-exercise-set}

Perform the indicated operations. no 27 \[(x+3)^2\]\textgreater{}\(& expand((x+3)^2)\)\(x^2+6\,x+9\)\(no 29\)\((y-5)^2\)\(\>\)\& expand((y-5)\^{}2) \[y^2-10\,y+25\]no 33 \[(2x+3y)^2\]\textgreater{}\(& expand((2\*x+3\*y)^2)\)\(9\,y^2+12\,x\,y+4\,x^2\)\(no 39\)\((3y+4)(3y-4)\)\(\>\)\& expand ((3*y+4)*(3*y-4)) \[9\,y^2-16\]no 42 \[(3x+5y)(3x-5y)\]\textgreater{}\(& expand ((3\*x+5\*y)\*(3\*x-5\*y))\)\(9\,x^2-25\,y^2\)\$ \#\# R.4 Exercise set

Faktor the trinomial no 23 \[t^2+8t+15\]\textgreater\$\& solve (t\^{}2+8*t+15)

\$\left[ t=-3 , t=-5 \right] \$\$

no 24. \[y^2+12y+27\]\textgreater\$\& solve(y\^{}2+12*y+27)

\$\left[ y=-9 , y=-3 \right] \$\$

Factor the difference of squares no 47 \[z^2-81\]\textgreater\$\& solve(z\^{}2-81)

\$\left[ z=-9 , z=9 \right] \$\$

no 48\[m^2-4\]\textgreater\$\& solve(m\^{}2-4)

\$\left[ m=-2 , m=2 \right] \$\$

no 49\[16x^2-9\]\textgreater\$\& solve(16*x\^{}2-9)

\$\left[ x=-\frac{3}{4} , x=\frac{3}{4} \right] \$\$

\section{R.5 Exercise set}\label{r.5-exercise-set}

no 31 Tentukan nilai x!\[7(3x+6)=11-(x+2)\]\textgreater\$\& solve(7*(3*x+6)=11-(x+2))

\$\left[ x=-\frac{3}{2} \right] \$\$

no 37 Tentukan nilai x!\[x^2+5x=0\]\textgreater\$\& solve(x\^{}2+5*x=0)

\$\left[ x=-5 , x=0 \right] \$\$

no 42 Tentukan nilai y!\[y^2+25=10y\]\textgreater\$\& solve(y\^{}2+25=10*y)

\$\left[ y=5 \right] \$\$

no 47 Tentukan nilai z!\[12z^2+z=6\]\textgreater\$\& solve(12*z\^{}(2)+z=6)

\$\left[ z=-\frac{3}{4} , z=\frac{2}{3} \right] \$\$

no 60 Tentukan nilai x!\[5x^2-75\]\textgreater\$\& solve(5*x\^{}2-75)

\$\left[ x=-\sqrt{15} , x=\sqrt{15} \right] \$\$

\section{R.6 Exercise set}\label{r.6-exercise-set}

Sederhanakan!

no 9

\textgreater\$\& ((x\textsuperscript{2-4)/(x}2-4*x+4)), \(&factor(%)
\)\(\frac{x+2}{x-2}\)\$\pandocbounded{\includegraphics[keepaspectratio]{images/EMT untuk Perhitungan Aljabar_Isni Azizah Utami_23030630016-161.png}}

no 10

\textgreater\$\& ((x\textsuperscript{2+2*x-3)/(x}2-9)), \(&factor(%)
\)\(\frac{x-1}{x-3}\)\$\pandocbounded{\includegraphics[keepaspectratio]{images/EMT untuk Perhitungan Aljabar_Isni Azizah Utami_23030630016-163.png}}

no 11

\textgreater\$\& ((x\textsuperscript{3-6*x}2+9*x)/(x\textsuperscript{3-3*x}2)), \(&factor(%)
\)\(\frac{x-3}{x}\)\$\pandocbounded{\includegraphics[keepaspectratio]{images/EMT untuk Perhitungan Aljabar_Isni Azizah Utami_23030630016-165.png}}

no 12

\textgreater\$\& ((y\textsuperscript{5-5*y}4+4*y\textsuperscript{3)/(y}3-6*y\^{}2+8*y)), \(&factor(%)
\)\(\frac{\left(y-1\right)\,y^2}{y-2}\)\$\pandocbounded{\includegraphics[keepaspectratio]{images/EMT untuk Perhitungan Aljabar_Isni Azizah Utami_23030630016-167.png}}

no 13

\textgreater\$\& ((6*y\textsuperscript{2+12*y-48)/(3*y}2-9*y+6)), \(&factor(%)
\)\(\frac{2\,\left(y+4\right)}{y-1}\)\$\pandocbounded{\includegraphics[keepaspectratio]{images/EMT untuk Perhitungan Aljabar_Isni Azizah Utami_23030630016-169.png}}

\section{REVIEW EXERCISE}\label{review-exercise}

Multiply. Assume that all exponents are integers.

no 70\[(x^n+10)(x^n-4)\]\textgreater{}\(&expand((x^n+10)\*(x^n-4))\)\(x^{2\,n}+6\,x^{n}-40\)\(no 71\)\((t^a+t^{-a})^2\)\(\>\)\& expand((t\textsuperscript{a+t}(-a))\^{}2) \[t^{2\,a}+\frac{1}{t^{2\,a}}+2\]no 72\[(y^b-z^c)(y^b+z^c)\]\textgreater{}\(& expand((y^b-z^c)\*(y^b+z^c))\)\(y^{2\,b}-z^{2\,c}\)\(no 72\)\((a^n-b^n)^3\)\(\>\)\& expand((a\textsuperscript{n-b}n)\^{}3) \[-b^{3\,n}+3\,a^{n}\,b^{2\,n}-3\,a^{2\,n}\,b^{n}+a^{3\,n}\]Factor.

no 74\[y^{2n}+16y^n+64\]\textgreater{}\(&factor(y^(2\*n)+16\*y^n+64)\)\(\left(y^{n}+8\right)^2\)\$Soal Chapter R test

no 32

Multiply and simplify

\textgreater{}\(& expand((x^2+x-6)\*(x^2+8\*x+15))\)\(x^4+9\,x^3+17\,x^2-33\,x-90\)\(\>\)\&((x\textsuperscript{2+x-6)*(x}2+8*x+15)), \(&factor(%)
\)\(\left(x-2\right)\,\left(x+3\right)^2\,\left(x+5\right)\)\$\pandocbounded{\includegraphics[keepaspectratio]{images/EMT untuk Perhitungan Aljabar_Isni Azizah Utami_23030630016-182.png}}

\section{2.3 Exercise Set}\label{exercise-set}

Diberikan fungsi\[f(x)=3x+1 , g(x)=x^2-2x-6 , h(x)=x^3\]Cari

Nomor 4. \[\left(g\circ h\right)\left(1/2\right)\]\textgreater\$\&x:=1/2; \$\&hx:=x\^{}3; \$\&gx:=hx\^{}2-2*hx-6; \(&gx\)\(-\frac{399}{64}\)\(Nomor 6.\)\(\ \left(f\circ g\right)\left(1/3\right)\)\(\>\)\&x:=1/3; \$\&gx:=x\^{}2-2*x-6; \$\&fx:=3*gx+1; \(&fx\)\(-\frac{56}{3}\)\(Nomor 9.\)\(\left(g\circ g\right)\left(-2\right)\)\$ \textgreater\$\&x:=-2; \$\&gx:=x\^{}2-2*x-6; \$\&gx:=gx\^{}2-2*gx-6; \(&gx\)\(-6\)\(Cari\)\(\left(f\circ g\right)\left(x\right) dan \left(g\circ f\right)\left(x\right)\)\$ dan domain nya !

Nomor 17. \[f(x)=x+3\ ,\ g(x)=x-3\]\[\left(f\circ g\right)\left(x\right)=\]

\textgreater\$\&gx:=x-3; \$\&fx:=gx+3; \(&fx\)\(-2\)\(dengan domainnya\)\(D_{f\circ g}=\left\{x\in\mathbb{R}\right\}\)\$ \[\left(g\circ f\right)\left(x\right)=\]\textgreater\$\&fx:=x+3; \$\&gx:=fx-3; \$\&gx

\[-2\]dengan domainnya\[D_{g\circ f}=\left\{x\in\mathbb{R}\right\}\]Nomor 23.\[f(x)=4/(1-5x)\ \ ,\ g(x)=1/x\] \[\left(f\circ g\right)\left(x\right)=\]\textgreater\$\&gx:=1/x; \$\&fx:=4/(1-5*gx); \(&fx\)\(\frac{8}{7}\)\(dengan domain\)\(D_{f\circ g}=\left\{x\in\mathbb{R}|x\neq 0\cup x\neq 5\right\}\)\$ \[\left(g\circ f\right)\left(x\right)=\]\textgreater\$\&fx:=4/(1-5*x); \$\&gx:=1/fx; \(&gx\)\(\frac{11}{4}\)\(dengan domainnya\)\(D_{g\circ f}=\left\{x\in\mathbb{R}\right\}\)\(dengan domainnya\)\(D_{g\circ f}=\left\{x\in\mathbb{R}\right\}\)\$ \#\# 3.1 Exercise set

Use the quadratic formula to find exact solutions.

no 39\[5m^2+3m=2\]\textgreater\$\&solve(5*m\^{}2+3*m=2)

\$\left[ m=\frac{2}{5} , m=-1 \right] \$\$

no 40\[2y^2-3y-2=0\]\textgreater\$\&solve(2*y\^{}2-3*y-2)

\$\left[ y=-\frac{1}{2} , y=2 \right] \$\$

no 42 \[3t^2+8t+3=0\]\textgreater\$\&solve(3*t\^{}2+8*t+3=0)

\$\left[ t=\frac{-\sqrt{7}-4}{3} , t=\frac{\sqrt{7}-4}{3} \right] \$\$

no 45\[5t^2-8t=3\]\textgreater\$\&solve(5*t\^{}2-8*t=3)

\$\left[ t=\frac{4-\sqrt{31}}{5} , t=\frac{\sqrt{31}+4}{5} \right] \$\$

no 48\[2t^2-5t=1\]\textgreater\$\&solve(2*t\^{}2-5*t=1)

\$\left[ t=\frac{5-\sqrt{33}}{4} , t=\frac{\sqrt{33}+5}{4} \right] \$\$

\section{3.4 Exercise set}\label{exercise-set-1}

solve

no 1\[\frac14+\frac15=\frac1t\]\textgreater\$\&solve((1/4)+(1/5)=(1/t))

\$\left[ t=\frac{20}{9} \right] \$\$

no 2\[\frac13-\frac56=\frac1x\]\textgreater\$\&solve((1/3)-(5/6)=(1/x))

\$\left[ x=-2 \right] \$\$

no 5\[\frac12+\frac2x=\frac13+\frac3x\]\textgreater\$\&solve((1/2)+(2/x)=(1/3)+(3/x))

\$\left[ x=6 \right] \$\$

no 35\[\sqrt{1-2x}=3\]\textgreater\$\&solve((1-2*x)\^{}(1/2)=3)

\$\left[ x=-4 \right] \$\$

no 36\[\sqrt{2-7x}=2\]\textgreater\$\& solve((2-7*x)\^{}(1/2)=2)

\$\left[ x=-\frac{2}{7} \right] \$\$

\section{3.5 Exercise set}\label{exercise-set-2}

\textgreater\&load(fourier\_elim)

\begin{verbatim}
       D:/Euler x64/maxima/share/maxima/5.35.1/share/fourier_elim/fou\
rier_elim.lisp
\end{verbatim}

no 44

\textgreater\$\& fourier\_elim({[}4*x{]}\textgreater20,{[}x{]}) //4*x\textgreater20

\$\left[ \left[ 4\,\left(x-5\right) \right] \textgreater0 \right{]} \$\$

no 45

\textgreater\$\&fourier\_elim({[}x+8{]}\textless9,{[}x{]})// x+8\textless9

\$\left[ \left[ 1-x \right] \textgreater0 \right{]} \$\$

no 47

\textgreater\$\&fourier\_elim({[}x+8{]}\textgreater= 9,{[}x{]})//x+8 \textgreater=9

\$\left[ \left[ x-1 \right] =0 \right{]} \lor \left[ \left[ x-1   \right] \textgreater0 \right{]} \$\$

no 52

\textgreater\$\&fourier\_elim({[}3*x+4{]}\textless13,{[}x{]})//3*x+4\textless13

\$\left[ \left[ -3\,\left(x-3\right) \right] \textgreater0 \right{]} \$\$

no 62

\textgreater\$\&fourier\_elim({[}3*x+5{]}\textless0,{[}x{]})//3*x+5\textless0

\$\left[ \left[ -3\,x-5 \right] \textgreater0 \right{]} \$\$

\section{Chapter 3 Test}\label{chapter-3-test}

no 8

\textgreater\$\&solve(3/(3*x+4)+2/(x-1)=2)

\$\left[ x=\frac{13}{6} , x=-1 \right] \$\$

no 9

\textgreater\$\&solve((sqrt(x+4))-2=1)

\$\left[ x=5 \right] \$\$

no 11

\textgreater\$\&fourier\_elim({[}x=4{]}=7,{[}x{]})//x+4=7

\$\left[ \left[ x-7=-3 \right] =0 \right{]} \$\$

no 13

\textgreater\$\&fourier\_elim({[}x+3{]}\textless=4,{[}x{]})//x+3\textless=4

\$\left[ \left[ x-1 \right] =0 \right{]} \lor \left[ \left[ 1-x   \right] \textgreater0 \right{]} \$\$

no 15

\textgreater\$\&fourier\_elim({[}x+5{]}\textgreater2,{[}x{]})//x+5\textgreater2

\$\left[ \left[ x+3 \right] \textgreater0 \right{]} \$\$

no 19

\textgreater\$\&solve(x\^{}2+4*x =1,x)

\$\left[ x=-\sqrt{5}-2 , x=\sqrt{5}-2 \right] \$\$

\section{4.1 Exercise set}\label{exercise-set-3}

no 23

Use substitution to determine whether 4,5,and-2 are zeros of\[f(x) = x^3-9x^2+14x+24\]\textgreater function f(x):= (x\textsuperscript{3-9*x}2+14*x+24) \textgreater f(4)

\begin{verbatim}
0
\end{verbatim}

\textgreater f(5)

\begin{verbatim}
-6
\end{verbatim}

\textgreater f(-2)

\begin{verbatim}
-48
\end{verbatim}

Jadi, hasil subtitusi akan 0 apabila mensubtitusikan nilai x=4.

no 24

Use substitution to determine whether 2, 3, and -1 are zeros of\[f(x)=2x^3-3x^2+x+6\]\textgreater function f(x):= (2*x\textsuperscript{3-3*x}2+x+6) \textgreater f(2)

\begin{verbatim}
12
\end{verbatim}

\textgreater f(3)

\begin{verbatim}
36
\end{verbatim}

\textgreater f(-1)

\begin{verbatim}
0
\end{verbatim}

Jadi, hasil subtitusi akan 0 apabila mensubtitusikan nilai x=-1.

no 25

Use subtitution to detimen whether 2,3,and -1 are zeros of\[g(x)=x^4-6x^3+8x^2+6x-9\]\textgreater function g(x):= (x\textsuperscript{4-6*x}3+8*x\^{}2+6*x-9) \textgreater g(2)

\begin{verbatim}
3
\end{verbatim}

\textgreater g(3)

\begin{verbatim}
0
\end{verbatim}

\textgreater g(-1)

\begin{verbatim}
0
\end{verbatim}

Jadi, hasil subtitusi akan 0 apabila mensubtitusikan nilai x=3 dan x=-1.

no 26

Use substitution to determine whether 1, -2, and 3 are zeros of\[g(x)= x^4-x^3-3x^2+5x-2\]\textgreater function g(x):= (x\textsuperscript{4-x}3-3*x\^{}2+5*x-2) \textgreater g(1)

\begin{verbatim}
0
\end{verbatim}

\textgreater g(-2)

\begin{verbatim}
0
\end{verbatim}

\textgreater g(3)

\begin{verbatim}
40
\end{verbatim}

Jadi, hasil subtitusi akan 0 apabila mensubtitusikan nilai x=-2 dan x=3.

Find the zeros of the polynomial function and state the multiplicity of each

no 37\[f(x)=x^4-4x^2+3\]\textgreater\$\&solve(x\textsuperscript{4-4*x}2+3,x)

\$\left[ x=-1 , x=1 , x=-\sqrt{3} , x=\sqrt{3} \right] \$\$

\section{4.3 Exercise set}\label{exercise-set-4}

no 1

for the function

\textgreater function f(x) \&= x\textsuperscript{4-6*x}3+x\^{}2+24*x-20

\begin{verbatim}
                       4      3    2
                      x  - 6 x  + x  + 24 x - 20
\end{verbatim}

use long division to determine whether each of the following is a factor of f(x).

a.\[x+1\] b.\[x-2\] c.\[x+5\]\textgreater{}\(&factor(x^4-6\*x^3+x^2+24\*x-20)\)\(\left(x-5\right)\,\left(x-2\right)\,\left(x-1\right)\,\left(x+2  \right)\)\$Jadi factor dari fungsi f(x) di atas adalah b. yaitu (x-2)

no 2

for the function

\textgreater function h(x) \&= x\textsuperscript{3-x}2-17*x-15

\begin{verbatim}
                          3    2
                         x  - x  - 17 x - 15
\end{verbatim}

use long division to determine whether each of the following is a factor of f(x).

a.\[x+5\] b.\[x+1\] c.\[x+3\]\textgreater{}\(&factor(x^3-x^2-17\*x-15)\)\(\left(x-5\right)\,\left(x+1\right)\,\left(x+3\right)\)\$Jadi factor dari fungsi f(x) di atas adalah b dan c yaitu (x+1) dan (x+3)

Factor the polynomial function f(x). Then solve the equation f(x)=0.

no 39

\textgreater function f(x)\&= x\textsuperscript{3+4*x}2+x-6

\begin{verbatim}
                           3      2
                          x  + 4 x  + x - 6
\end{verbatim}

\textgreater{}\(&factor(x^3+4\*x^2+x-6)\)\(\left(x-1\right)\,\left(x+2\right)\,\left(x+3\right)\)\(\>\)\&solve(x\textsuperscript{3+4*x}2+x-6=0,x)

\$\left[ x=-3 , x=-2 , x=1 \right] \$\$

no 40

\textgreater function f(x)\&= x\textsuperscript{3+5*x}2-2*x-24

\begin{verbatim}
                          3      2
                         x  + 5 x  - 2 x - 24
\end{verbatim}

\textgreater{}\(&factor(x^3+5\*x^2-2\*x-24)\)\(\left(x-2\right)\,\left(x+3\right)\,\left(x+4\right)\)\(\>\)\&solve(x\textsuperscript{3+5*x}2-2*x-24)

\$\left[ x=-4 , x=-3 , x=2 \right] \$\$

no 41

\textgreater function f(x) \&= x\textsuperscript{3-6*x}2+3*x+10

\begin{verbatim}
                          3      2
                         x  - 6 x  + 3 x + 10
\end{verbatim}

\textgreater{}\(&factor(x^3-6\*x^2+3\*x+10)\)\(\left(x-5\right)\,\left(x-2\right)\,\left(x+1\right)\)\(\>\)\&solve(x\textsuperscript{3-6*x}2+3*x+10)

\$\left[ x=2 , x=-1 , x=5 \right] \$\$

\section{Mid-Chapter Mixed Review}\label{mid-chapter-mixed-review}

Use synthetic division to find the function values.

no 18\[g(x) = x^3-9x^2+4x-10\] find g(-5)

\textgreater function g(x) \&= x\textsuperscript{3-9*x}2+4*x-10

\begin{verbatim}
                          3      2
                         x  - 9 x  + 4 x - 10
\end{verbatim}

\textgreater g(-5)

\begin{verbatim}
-380
\end{verbatim}

no 19\[f(x)= 20x^2-40x\]

find f(1/2)

\textgreater function f(x) \&= 20*x\^{}2-40*x

\begin{verbatim}
                                 2
                             20 x  - 40 x
\end{verbatim}

\textgreater f(1/2)

\begin{verbatim}
-15
\end{verbatim}

no 20\[f(x) = 5x^4+x^3-x\]

find\[f(-\sqrt2)\]\textgreater function f(x)\&= 5*x\textsuperscript{4+x}3-x

\begin{verbatim}
                               4    3
                            5 x  + x  - x
\end{verbatim}

\textgreater f(-(2)\^{}(1/2))

\begin{verbatim}
18.5857864376
\end{verbatim}

Factor the polynomial function f(x). Then solve the equation f(x)=0.

no 23

\textgreater function h(x) \&= x\textsuperscript{3-2*x}2-55*x+56

\begin{verbatim}
                         3      2
                        x  - 2 x  - 55 x + 56
\end{verbatim}

\textgreater{}\(&factor(x^3-2\*x^2-55\*x+56)\)\(\left(x-8\right)\,\left(x-1\right)\,\left(x+7\right)\)\(\>\)\&solve(x\textsuperscript{3-2*x}2-55*x+56=0)

\$\left[ x=-7 , x=8 , x=1 \right] \$\$

no 24

\textgreater function g(x) \&= x\textsuperscript{4-2*x}3-13*x\^{}2+14*x+24

\begin{verbatim}
                     4      3       2
                    x  - 2 x  - 13 x  + 14 x + 24
\end{verbatim}

\textgreater{}\(&factor(x^4-2\*x^3-13\*x^2+14\*x+24)\)\(\left(x-4\right)\,\left(x-2\right)\,\left(x+1\right)\,\left(x+3  \right)\)\(\>\)\&solve(x\textsuperscript{4-2*x}3-13*x\^{}2+14*x+24)

\$\left[ x=-3 , x=-1 , x=2 , x=4 \right] \$\$

\backmatter
\end{document}
